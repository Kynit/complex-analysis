\documentclass{article}

\usepackage{amsmath}
\usepackage{parskip}
\usepackage{tikz}
\renewcommand{\emph}{\textbf}
\renewcommand{\bar}{\overline}
\DeclareMathOperator{\Log}{Log}
\DeclareMathOperator{\Arg}{Arg}
\DeclareMathOperator{\sech}{sech}
\DeclareMathOperator{\csch}{csch}
\DeclareMathOperator{\Res}{Res}

\begin{document}

\section{Frame 68 -- Isolated Singular Points}
\subsection{Definition}
Earlier, we defined a \emph{singular point} of a function $f$ as a point $z_0$ where $f$ is not analytic, but $f$ is analytic at some point in every neighbourhood of $z_0$. Additionally, we will define an \emph{isolated} singular point as such a point where there exists a deleted neighbourhood on which $f$ is analytic.

\subsection{Examples}
\textit{The function
\[
	f(z) = \frac{z+1}{z^3 (z^2 + 1)}
\]
has three singular points at $z = 0$ and $z = \pm i$. These three points are isolated singularities.}

\textit{The principal branch of the logarithm
\[
	\Log z = \ln r + i\Theta 	\quad (-\pi < \theta < \pi)
\]
has a singular point at $z = 0$. However, this is not an isolated singular point, since every neighbourhood of $0$ also contains some points on the negative real axis, where the function is not analytic.}

\textit{The function
\[
	f(z) = \frac{1}{\sin(\pi / z)}
\]
has singular points at $z = 0$ and $z = 1/n$ for all integers $n$. All of these singular points are isolated except for the one at $z = 0$ -- every neighbourhood of $0$ also contains a point $z = 1/m$ because we can find such a point
\[
	0 < 1/m < \epsilon
\]
for each $\epsilon$.}

\subsection{Important Points}
Note that if a function has a finite number of singular points, then all of these must be isolated -- we can make a neighbourhood around each singular point that does not contain any others.

Finally, note that we may also refer to the point at infinity as an isolated singular point. This happens if there exists a positive $R_1$ such that there are no singularities in the region
\[
	R_1 < |z| < \infty
\]


\clearpage
\section{Frame 69 -- Residues}
\subsection{Definition}
Suppose that $f$ is a function with an isolated singular point at $z_0$. Then, there exists some deleted neighbourhood
\[
	0 < |z - z_0| < R_2
\]
where $f$ is analytic. On this domain, we can express $f(z)$ as the Laurent series
\[
	f(z) = \sum_{n=0}^\infty a_n (z - z_0)^n
	+ \frac{b_1}{z - z_0}
	+ \frac{b_2}{(z - z_0)^2}
	+ \dots
	+ \frac{b_n}{(z - z_0)^n}
	+ \dots
\]
These coefficients come from various integral representations. In particular,
\[
	b_n = \frac{1}{2\pi i} \int_C \frac{f(z)}{(z - z_0)^{-n+1}} dz
\]
where $C$ is a positively oriented, simple, closed contour around $z_0$ in the deleted neighbourhood described above. When $n = 1$, this expression becomes
\[
	\int_C f(z) dz = 2\pi i b_1
\]

We say that the complex number $b_1$ is called the \emph{residue} of $f$ at the isolated singular point $z_0$, and we write
\[
	b_1 = \Res_{z = z_0} f(z)
\]
so the contour integral around $z_0$ becomes
\[
	\int_C f(z) dz = 2\pi i \Res_{z = z_0} f(z)
\]

\subsection{Examples}
\textit{Example 1: We can evaluate the integral
\[
	\int_C z^2 \sin\left( \frac{1}{z} \right) dz
\]
where $C$ is the positively-oriented unit circle. First, we note that the integrand is analytic everywhere except the origin, so the Laurent series converges on $0 < |z| < \infty$. Then, we can write
\begin{align*}
	z^2 \sin(1 / z)
	&= z^2 \sum_{n=0}^\infty \frac{(-1)^n}{(2n+1)!} (1 / z)^{2n+1} \\
	&= \sum_{n=0}^\infty \frac{(-1)^n}{(2n+1)!} z^{1 - 2n}
\end{align*}
so the coefficient $b_1$ is $-1/3!$. Thus,
\[
	\int_C z^2 \sin\left( \frac{1}{z} \right) dz
	= 2\pi i \left( -\frac{1}{3!} \right) 
	= -\frac{\pi i}{3}
\]}

\textit{Example 2: We can repeat the previous problem for the integral
\[
	\int_C e^{1 / z^2}
\]
Since this series is
\begin{align*}
	e^{1 / z^2}
	&= \sum_{n=0}^\infty \frac{(1 / z^2)^n}{n!} \\
	&= \sum_{n=0}^\infty \frac{1}{n! z^{2n}} \\
	&= 1 + \frac{1}{z^2} + \frac{1}{2 z^4} + \dots 
\end{align*}
the residue at $0$ is $0$, so
\[
	\int_C e^{1 / z^2} = 0
\]}

\textit{Example 3: Finally, we can use residues to evaluate the integral
\[
	\int_C \frac{1}{z(z - 2)^4} dz
\]
around the positively-oriented circle $|z - 2| = 1$. The Laurent series is
\begin{align*}
	\frac{1}{z(z - 2)^4}
	&= \frac{1}{(z - 2)^4} \frac{1}{2 + (z - 2)} \\
	&= \frac{1}{2(z - 2)^4} \frac{1}{1 - \frac{-(z - 2)}{2}} \\
	&= \frac{1}{2(z - 2)^4} \sum_{n=0}^\infty \frac{(-1)^n}{2^n} (z - 2)^n \\
	&= \sum_{n=0}^\infty \frac{(-1)^n}{2^{n+1}} (z - 2)^{n-4}
\end{align*}
Thus, the coefficient of $1 / (z - 2)$ is at $n = 3$, and
\[
	b_1 = -\frac{1}{16}
\]
so
\[
	\int_C \frac{1}{z(z - 2)^4} dz = 2\pi i\left(-\frac{1}{16}\right) 
	= -\frac{\pi i}{8}
\]}


\clearpage
\section{Frame 70 -- Cauchy's Residue Theorem}
\subsection{Theorem}
If a function has a finite number of singularities inside a contour, then we can use residues to simplify these contour integrals. The following theorem is Cauchy's residue theorem:

\textit{Theorem: Suppose that $C$ is a positively-oriented, simple, closed contour. If a function $f$ is analytic on $C$ and has a finite number of singularities at $z_k$ (where $k = 1, 2, \dots, n$) inside $C$, then
\[
	\int_C f(z) dz = 2\pi i \sum_{k=1}^\infty \Res_{z = z_k} f(z)
\]}

Proof: suppose that the contours $C_k$ are small, positively-oriented circles centered at the points $z_k$. Then, the region $C - \sum C_k$ is a multiply connected domain over which $f$ is analytic, so
\[
	\int_C f(z)~dz - \sum_{k=1}^\infty \int_{C_k} f(z)~dz = 0
\]
Then, since
\[
	\int_{C_k} f(z)~dz = 2\pi i \Res_{z = z_k} f(z)
\]
the integral can be written as in the theorem.

\subsection{Example}
If $C$ is the counterclockwise circular contour $|z| = 2$, then we can evaluate the integral
\[
	\int_C \frac{5z - 2}{z(z - 1)} dz
\]
using Cauchy's residue theorem. First, we can find the residue $B_1$ at $z = 0$ by writing out the function as the series
\begin{align*}
	\frac{5z - 2}{z(z - 1)}
	&= \frac{5z - 2}{z} \cdot \frac{-1}{1 - z} \\
	&= \left( 5 - \frac{2}{z} \right) (-1 - z - z^2 - \dots) \\
	&= \frac{2}{z} - 3 - 3z - \dots
\end{align*}
so $B_1 = 2$. Then, we can find the residue $B_2$ at $z = 1$ by writing
\begin{align*}
	\frac{5z - 2}{z(z - 1)}
	&= \frac{5(z - 1) + 3}{z - 1} \cdot \frac{1}{1 + (z - 1)} \\
	&= \left( 5 + \frac{3}{z - 1} \right) (1 - (z-1) + (z-1)^2 - \dots) \\
	&= \frac{3}{z - 1} + 2 - 2(z - 1) + \dots
\end{align*}
so $B_2 = 3$, and
\[
	\int_C \frac{5z - 2}{z(z - 1)}
	= 2\pi i(2 + 3)
	= 10\pi i
\]

Alternatively, we could use partial fractions to write
\[	
	\frac{5z - 2}{z(z - 1)} = \frac{2}{z} + \frac{3}{z-1}
\]
and immediately discover that $B_1 = 2$ and $B_2 = 3$.


\clearpage
\section{Frame 71 -- Residue at Infinity}
\subsection{Definition}
Next, suppose that a function $f$ is analytic everywhere in the plane except for a finite number of singular points and $C$ is a positively oriented, simple, closed contour containing all of these points. Then, we can make a circle containing all of these points ($|z| = R_1$), and $f$ must be analytic in the domain $R_1 < |z| < \infty$. 

We can define the residue at infinity by making another circular contour, this one in the \emph{negative direction} (ie: keeping infinity on the left), with a radius $R_0 > R_1$. Then, the \emph{residue at infinity} is
\[
	\int_{C_0} f(z)~dz = 2\pi i \Res_{z = \infty} f(z)
\]
To simplify this integral, we know that
\[
	\int_C f(z)~dz = \int_{-C_0} f(z)~dz = -\int_{C_0} f(z)~dz
\]
so
\[
	\int_C f(z)~dz = -2\pi i \Res_{z = \infty} f(z)
\]

\subsection{Main Theorem}
We can find the residue at infinity by writing
\[
	f(z) = \sum_{n = -\infty}^\infty c_n z^n	\quad	(R_1 < |z| < \infty)
\]
where
\[
	c_n = \frac{1}{2\pi i} \int_{-C_0} \frac{f(z)}{z^{n+1}} dz
\]
Then, if we modify the series to represent $f(1/z) / z^2$, we can write
\[
	\frac{1}{z^2} f\left(\frac{1}{z} \right)
	= \sum_{n=-\infty}^\infty \frac{c_n}{z^{n+2}}
	= \sum_{n=-\infty}^\infty \frac{c_{n-2}}{z^n}
\]
so the residue of this modified function at zero is nearly the same as the original residue at infinity. Symbolically,
\[
	\Res_{z = \infty} f(z) = -2\pi i \Res_{z = 0} \left[ \frac{1}{z^2} f\left(\frac{1}{z} \right) \right]
\]

This result is shown in the following theorem:

\textit{Theorem: If a function $f$ is analytic everywhere except a finite number of singular points, and all of these points are interior to $C$, then
\[
	\int_C f(z)~dz = 2\pi i \Res_{z = 0} \left[ \frac{1}{z^2} f\left( \frac{1}{z} \right)\right]
\]}

\subsection{Example}
We can now re-do the problem from the previous section. Since all of the singular points were contained inside $|z| = 2$, we can write
\begin{align*}
	\frac{1}{z^2} f\left( \frac{1}{z} \right)
	&= \frac{1}{z^2} \frac{5/z - 2}{1/z \cdot (1/z - 1)} \\
	&= \frac{5 - 2z}{z(1 - z)} \\
	&= \left( \frac{5}{z} - 2 \right) (1 + z + z^2 + \dots) \\
	&= \frac{5}{z} + 3 + 3z + \dots
\end{align*}
so the residue at infinity is $5$, and
\[
	\int_C \frac{5z - 2}{z(z - 1)} dz = 2\pi i(5) = 10\pi i
\]


\clearpage
\section{Frame 72 -- The Three Singular Points}
We saw in the previous section that a function $f$ with an isolated singular point at $z_0$ can be written as the Laurent series
\[
	f(z) = \sum_{n=0}^\infty a_n (z - z_0)^n
	+ \frac{b_1}{z - z_0}
	+ \frac{b_2}{(z - z_0)^2} + \dots
	+ \frac{b_n}{(z - z_0)^n} + \dots
\]
on the domain $0 < |z - z_0| < R$. We refer to the negative powers as the \emph{principal part} of $f$ at $z_0$. We can use the qualities of this principal part to classify a singular point into one of three categories.

\subsection{Poles}
If the principal part has at least one non-zero term but a finite number of terms, as in
\[
	f(z) = \sum_{n=0}^\infty a_n (z - z_0)^n
	+ \frac{b_1}{z - z_0}
	+ \frac{b_2}{(z - z_0)^2} + \dots
	+ \frac{b_m}{(z - z_0)^m}
\]
then the isolated singular point is referred to as a \emph{pole of order $m$}. A first-order pole is called a \emph{simple pole}.

\textit{Example 1: the function
\[
	\frac{z^2 - 2z + 3}{z - 2}
	= \frac{z(z-2) + 3}{z - 2}
	= 2 + (z - 2) + \frac{3}{z - 2}
\]
has a simple pole at $z = 2$. Its residue there is $3$.}

\textit{Example 2: the function
\[
	\frac{1}{z^2(1 + z)}
	= \frac{1}{z^2}(1 - z + z^2 - z^3 + \dots)
	= \frac{1}{z^2} - \frac{1}{z} + 1 - z + \dots
\]
has a pole of order $2$ at the origin. Its residue there is $-1$.}

\textit{Example 3: the function
\[
	\frac{\sinh z}{z^4}
	= \frac{1}{z^4} \left(z + \frac{z^3}{3!} + \frac{z^5}{5!} + \dots \right)
	= \frac{1}{z^3} + \frac{1}{3! \cdot z} + \frac{z}{5!} + \dots
\]
has a third-order pole at the origin with a residue of $1/6$.}

\subsection{Removable Singular Points}
If every term of the principal part is zero, then we say that $z_0$ is a \emph{removable singular point}. This allows us to (re)define $f$ at the point $z_0$ as
\[
	f(z_0) = a_0
\]
so that the function becomes defined over the non-deleted disk, removing the singularity. Note that the residue at any removable singular point is zero.

\textit{Example 4: we can write
\begin{align*}
	\frac{1 - \cos z}{z^2}
	&= \frac{1}{z^2} \left[ 1 - \left(1 - \frac{z^2}{2!} + \frac{z^4}{4!} - \frac{z^6}{6!} + \dots \right) \right] \\
	&= \frac{1}{z^2} \left[ \frac{z^2}{2!} - \frac{z^4}{4!} + \frac{z^6}{6!} - \dots \right] \\
	&= \frac{1}{2} - \frac{z^2}{4!} + \frac{z^4}{6!}
\end{align*}
so if we assign the value $f(0) = 1/2$, then the function becomes entire.}

\subsection{Essential Singular Points}
If the principal part has an infinite number of non-zero terms, we say that $z_0$ is an \emph{essential singular point} of $f$.

\textit{Example 5: the function
\[
	e^{1/z}
	= \sum_{n=0}^\infty \frac{1}{n! \cdot z^n}
	= 1 + \frac{1}{z} + \frac{1}{2! \cdot z^2} + \dots
\]
has an essential singular point at the origin, with a residue of $1$.}

We can also briefly state \emph{Picard's theorem}: \textit{In every neighbourhood of an essential singular point, a function assumes every finite value (except possibly zero) an infinite number of times.}


\end{document}