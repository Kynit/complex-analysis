\documentclass{article}

\usepackage{amsmath}
\usepackage{parskip}
\usepackage{tikz}
\renewcommand{\emph}{\textbf}
\renewcommand{\bar}{\overline}
\DeclareMathOperator{\Log}{Log}
\DeclareMathOperator{\Arg}{Arg}
\DeclareMathOperator{\sech}{sech}
\DeclareMathOperator{\csch}{csch}

\begin{document}

\section{Frame 37 -- Derivatives with Real Variables}
\subsection{Definition}
In the previous chapter, we looked at derivatives of complex functions of a complex variable $z$. Now, we look at the derivatives of a complex-valued function of a real variable $t$. If we write our function as
\[
	w(t) = u(t) + iv(t)
\]
where $u$ and $v$ are real-valued, then we can define the derivative of $w$ at a point $t$ as
\[
	w'(t) = \frac{d}{dt} w(t) = u'(t) + iv'(t)
\]
provided that $u'$ and $v'$ exist at $t$.

\subsection{Properties}
If $z_0 = x_0 + iy_0$ is a complex constant, then we can show that
\begin{align*}
	\frac{d}{dt} [z_0 w(t)]
	&= [(x_0 + iy_0)(u(t) + iv(t)]' \\
	&= [x_0 u(t) - y_0 v(t)]' + i[y_0 u(t) + x_0 v(t)]' \\
	&= [x_0 u'(t) - y_0 v'(t)] + i[y_0 u'(t) + x_0 v'(t)] \\
	&= z_0 w'(t)
\end{align*}
as we expect.

Next, if $z_0$ is still a complex constant, the derivative of $e^{z_0 t}$ is
\begin{align*}
	\frac{d}{dt} e^{z_0 t} 
	&= \frac{d}{dt} e^{x_0 t}(\cos y_0 t + i sin y_0 t) \\
	&= \frac{d}{dt} e^{x_0 t} \cos y_0 t + i \frac{d}{dt} e^{x_0 t} \sin y_0 t \\
	&= (x_0 + iy_0)(e^{x_0 t} \cos y_0 t + ie^{x_0 t} \sin y_0 t) \\
	&= z_0 e^{z_0 t}
\end{align*}

Many other rules carry over from standard calculus. However, some rules no longer apply. For instance, in calculus, the mean value theorem for derivatives states that
\[
	w'(c) = \frac{w(b) - w(a)}{b - a}
\]
for some $c$ in the interval $a \le c \le b$ as long as $w$ is continuous. However, this is easily disproved by the function
\[
	w(t) = e^{it}
\]
If $a = 0$ and $b = 2\pi$, then $w(a) = w(b) = 1$ and we expect to find a point $c$ in $[0, 2\pi]$ such that $w'(c) = 0$. However, no such points exist -- the magnitude of the derivative is always $1$.


\clearpage
\section{Frame 38 -- Definite Integrals of Complex Functions}
\subsection{Definitions}
If $w(t)$ is a complex-valued function of a real variable $t$, as in the previous section
\[
	w(t) = u(t) + iv(t)
\]
then we define the \emph{definite integral} of $w(t)$ over the interval $a \le t \le b$ as
\[
	\int_a^b w(t) dt = \int_a^b u(t) dt + i \int_a^b v(t) dt
\]
provided the two right-side integrals exist. Then,
\begin{align*}
	\Re\left[\int_a^b w(t)dt\right] &= \int_a^b \Re[w(t)] dt \\
	\Im\left[\int_a^b w(t)dt\right] &= \int_a^b \Im[w(t)] dt
\end{align*}
Improper integrals over unbounded intervals are defined similarly.

The two real integrals will exist as long as $u$ and $v$ are \emph{piecewise continuous} on the interval $[a, b]$ -- that is, continuous everywhere in the interval except possibly for a finite number of points where it has one-sided limits. When $u$ and $v$ are piecewise continuous, we say that $w$ is also piecewise continuous.

\subsection{Properties}
The most common rules of integrals from calculus apply here as well:
\begin{itemize}
	\item $\int z_0 w(t)dt = z_0 \int w(t)$
	
	\item $\int w_1(t) + w_2(t)dt = \int w_1(t)dt + \int w_2(t)dt$
	
	\item $\int_a^b w(t) dt = -\int_b^a w(t) dt$
	
	\item $\int_a^b w(t) dt = \int_a^c w(t) dt + \int_c^b w(t) dt$
\end{itemize}

We can also extend the fundamental theorem of calculus to complex integrals. Suppose that two functions
\begin{align*}
	w(t) &= u(t) + iv(t) \\
	W(t) &= U(t) + iV(t) 
\end{align*}
are continuous on the interval $[a, b]$ and $W'(t) = w(t)$ when $a \le t \le b$. Then, we can write
\[
	\int_a^b w(t) dt = W(b) - W(a) = W(t) \Big|_a^b 
\] 

\textit{Example: noting that the derivative of $\frac{1}{i} e^{it}$ is
\[
	\frac{d}{dt} \left( \frac{1}{i} e^{it} \right)
	= \frac{1}{i} ie^{it}
	= e^{it}
\]
we can evaluate $\int e^{it} dt$ as
\begin{align*}
	\int_0^{\pi/4} e^{it} dt 
	&= \frac{e^{it}}{i} \Big|_0^{\pi/4} \\
	&= \frac{1}{i} \left[ e^{\pi/4} - 1 \right] \\
	&= \frac{1}{i} \left[ \frac{1}{\sqrt{2}} - 1 + \frac{i}{\sqrt{2}} \right] \\
	&= \frac{1}{\sqrt{2}} + i\left( 1 - \frac{1}{\sqrt{2}} \right)
\end{align*}}

As in the previous section, the mean value theorem for integrals does not apply. We can show this by finding the integral $\int_0^{2\pi} e^{it} dt = 0$, even though the function is never zero on this interval.


\clearpage
\section{Frame 39 -- Contours}
\subsection{Definitions}
In calculus, integrals are defined on intervals of the real line. In complex analysis, we instead use curves in the complex plane.

An \emph{arc} is a set of points $z = (x, y)$ in the complex plane such that the functions
\[
	x = x(t),	\quad y = y(t);	\quad z = z(t) = x(t) + iy(t)
\]
are continuous functions of the parameter $t$, where $a \le t \le b$. This definition is a continuous mapping of the interval $a \le t \le b$ into the $z$ plane.

We say that an arc is \emph{simple} if it does not cross itself; ie:
\[
	z(t_1) \ne z(t_2)	\quad \text{for all } t_1 \neq t_2
\]
If a simple arc starts and ends at the same point ($z(a) = z(b)$), it is called a \emph{simple closed curve}. These curves are \emph{positively oriented} when they are oriented in the counterclockwise direction. 

\textit{Example: the unit circle
\[
	z = e^{i\theta}
\]
where $0 \le \theta \le 2\pi$ is a positively oriented simple closed curve centered at the origin with a radius of $1$. A more general circle is
\[
	z = z_0 + Re^{i\theta}
\]
which is centered at $z_0$ and has a radius of $R$.}

\subsection{Uniqueness}
Note that the parametric representation for any arc is not unique. If we know a function $\phi$ such that
\[
	t = \phi(\tau)
\]
maps the interval $\alpha \le \tau \le \beta$ onto the interval $a \le t \le b$. Then, the two equations
\[
	z(t) \quad (a \le t \le b)
\]
and
\[
	z(\phi(t)) \quad	(\alpha \le t \le \beta)
\]
represent the same arc.

\subsection{Smoothness}
Suppose that the real and imaginary components of $z$ are differentiable, and their derivatives are continuous. Then, the arc $z(t)$ is a \emph{differentiable arc}, and
\[
	|z'(t)| = |x'(t) + iy'(t)| = \sqrt{[x'(t)]^2 + [y'(t)]^2}
\]
is integrable. This allows us to find the length of an arc as
\[
	L = \int_a^b |z'(t)| dt
\]

If an arc is differentiable and $z'(t)$ is never zero (except maybe at $t = a$ or $t = b$), then we call the arc a \emph{smooth arc}. We can write the unit tangent vector
\[
	\mathbf{T} = \frac{z'(t)}{|z'(t)|}
\]
which has an angle of inclination of $\arg z'(t)$. 

A \emph{contour} is an arc which consists of a finite number of smooth arcs joined together. Specifically, if $z(t)$ represents a contour, then $z(t)$ is continuous and $z'(t)$ is piecewise continuous. If a contour is also a simple closed arc, we call it a \emph{simple closed contour}. 

The points on a simple closed arc are the boundary points of two different domains:
\begin{itemize}
	\item The interior of the arc, which is bounded;
	\item The exterior of the arc, which is unbounded.
\end{itemize}


\clearpage
\section{Frame 40 -- Contour Integrals}
\subsection{Definitions and conditions}
We can now integrate a complex function $f$ along a contour $C$, which starts and ends at points $z_1$ and $z_2$, respectively. This is effectively a line integral. These integrals can be written as
\[
	\int_C f(z) dz
\]
or, if the integral does not depend on the path taken,
\[
	\int_{z_1}^{z_2} f(z) dz
\]

This integral (along a complex path) represents an integral with respect to a real parameter $t$. If the contour $C$ is written as $z(t)$ on the interval $a \le t \le b$, then the integral represented is
\[
	\int_C f(z) dz = \int_a^b f[z(t)] z'(t) dt
\]
Since $z'(t)$ must be piecewise continuous, this integral exists as long as $f[z(t)]$ is also piecewise continuous on this interval.

\subsection{Basic properties}
From the definition and the properties of integrals, we can write
\[
	\int_C z_0 f(z) dz = z_0 \int_C f(z) dz
\]
and
\[
	\int_C [f(z) + g(z)] dz = \int_C f(z) dz + \int_C g(z) dz
\]
We can also create a new contour $-C$ that consists of the points in $C$ in reversed order -- this contour extends from $z_2$ to $z_1$. Integrating along this reversed contour, we find that
\begin{align*}
	\int_-C f(z) dz 
	&= \int_{-b}^{-a} f[z(-t)] \frac{d}{dt} z(-t) dt \\
	&= -\int_{-b}^{-a} f[z(-t)] z'(-t) dt \\
	&= -\int_a^b f[z(t)] z'(t) dt \\
	&= -\int_C f(z) dz
\end{align*}
We can also split up a contour $C$ into multiple legs $C_1, C_2, \dots$. If we can write a contour this way, then we say that $C = C_1 + C_2$. The contour integral along $C$ can then be written as
\[
	\int_C f(z) dz = \int_{C_1} f(z) dz + \int_{C_2} f(z) dz
\]


\clearpage
\section{Frame 41 -- Examples of Contour Integrals}
This section will show several specific examples of contour integrals.

\subsection{Example 1}
Suppose that the contour $C$ is the right hand half of the circle $|z| = 2$:
\[
	z = 2e^{i\theta},	\quad \left( -\frac{\pi}{2} \le \theta \le \frac{\pi}{2}\right)
\]
Then,
\begin{align*}
	\int_C \bar{z} dz
	&= \int_{-\pi/2}^{\pi/2} \bar{2e^{i\theta}} (2e^{i\theta})' d\theta \\
	&= 4i\int_{-\pi/2}^{\pi/2} e^{-i\theta} e^{i\theta} d\theta \\
	&= 4i\int_{-\pi/2}^{\pi/2} d\theta \\
	&= 4\pi i
\end{align*}
Also, note that all of the points on this semicircle satisfy
\[
	z \bar{z} = |z|^2 = 4
\]
so we can see from this result that
\[
	\int_C \frac{1}{z} dz = \pi i
\]

\subsection{Example 2}
Suppose that the points $O$, $A$, and $B$ are $0$, $i$, and $1 + i$, respectively. Then, if $C_1$ is the polyline $OAB$ and
\[
	f(z) = y - x - i3x^2
\]
then the contour integral of $f$ along $C_1$ is
\begin{align*}
	\int_{C_1} f(z) dz
	&= \int_{OA} f(z) dz + \int_{AB} f(z) dz \\
	&= \int_0^1 yi dy + \int_0^1 (1 - x - i3x^2) dx \\
	&= \frac{i}{2} + \int_0^1 (1 - x)dx - 3i\int_0^1 x^2 dx \\
	&= \frac{i}{2} + \frac{1}{2} - i \\
	&= \frac{1 - i}{2}
\end{align*}

Next, if $C_2$ is the line $OB$, the contour integral along this curve is
\begin{align*}
	\int_{C_2} f(z) dz 
	&= \int_0^1 -i3x^2(1 + i)dx \\
	&= 3(1 - i) \int_0^1 x^2 \\
	&= 1 - i
\end{align*}

Finally, the integral of $f$ over the simple closed contour $OABO$ is $C_1 - C_2$, which is
\[
	\int_{OABO} f(z) dz = \frac{-1 + i}{2}
\]

\subsection{Example 3}
Suppose that $C$ is any arbitrary smooth arc from a fixed point $z_1$ to another point $z_2$:
\[
	z = z(t)	\quad (a \le t \le b)
\]
The contour integral of $f(z) = z$ along this curve is
\begin{align*}
	\int_C z dz 
	&= \int_a^b z(t) z'(t) dt \\
	&= \int_a^b \frac{d}{dt} \frac{[z(t)]^2}{2} dt \\
	&= \frac{[z(t)]^2}{2} \Big|_a^b \\
	&= \frac{z_2^2 - z_1^2}{2}
\end{align*}
Note that this integral only depends on the endpoints of $C$ and not the path. This lets us write
\[
	\int_{z_1}^{z_2} z dz = \frac{z_2^2 - z_1^2}{2}
\]
This holds when $C$ is not a smooth contour. Since all contours are sums of finite numbers of smooth arcs, this expression holds for each arc in $C$, leading to the same final expression.

Also, note that the integral of $f(z) = z$ around any closed contour in the plane is zero.


\clearpage
\section{Frame 42 -- Examples with Branch Cuts}
A contour integral's path can include a point on a branch cut. The following two examples show this.

\subsection{Example 1}
Suppose we want to integrate the function 
\[
	f(z) = z^{1/2} = e^{\frac{1}{2} \log z}	\quad (0 < \arg z < 2\pi)
\]
on the semicircle
\[
	z = 3e^{i\theta}	\quad(0 \le \theta \le \pi)
\]
Although the function is not defined at $\theta = 0$, we can still write
\[
	f[z(\theta)] = e^{\frac{1}{2}(\ln 3 + i\theta)} = \sqrt{3} e^{i\theta/2}
\]
and the right hand limit of this function exists at $\theta = 0$. Thus, the integrand exists as long as we define the missing point as
\[
	f[z(0)] z'(0) = i3\sqrt{3}
\]
Then,
\begin{align*}
	\int_C f(z) dz 
	&= 3\sqrt{3} \int_0^\pi e^{i3\theta / 2} \\
	&= 3\sqrt{3} \frac{2}{3i} e^{i3\theta / 2} \Big|_0^\pi \\
	&= -\frac{2}{3i} (1 + i) \\
	&= -2\sqrt{3}(1 + i)
\end{align*}


\subsection{Example 2}
Suppose that we want to integrate the function
\[
	f(z) = z^{a - 1} = e^{(a - 1) \Log z} \quad (-\pi < \Arg z < \pi)
\]
on the positively oriented circle
\[
	z = Re^{i\theta}	\quad (-\pi \le \theta \le \pi)
\]
The contour integral is
\begin{align*}
	\int_C z^{a - 1} dz
	&= \int_{-\pi}^{\pi} iR^ae^{ia\theta} d\theta \\
	&= iR^a \int_{-\pi}^{\pi} e^{ia\theta} d\theta \\
	&= iR^a \left( \frac{e^{ia\theta}}{ia} \right)_{-\pi}^{\pi} \\
	&= i\frac{2R^a}{a} \frac{e^{ia\pi} - e^{-ia\pi}}{2i} \\
	&= i\frac{2R^a}{a} \sin a\pi
\end{align*}
Note that if $a$ is a non-zero integer, this integral is zero; if $a = 0$, this integral reduces to
\[
	\int_C \frac{dz}{z} = 2\pi i
\]


\clearpage
\section{Frame 43 -- Upper Bounds for Contour Integrals}
We can put a bound on the modulus of a contour integral by observing that
\[
	\left| \int_a^b w(t) dt \right| 
	\le \int_a^b \left| w(t) \right| dt
\]

\subsection{Theorem}
Suppose that $C$ is a contour with a length of $L$ and that $f(z)$ is a function that is piecewise continuous on $C$. If $M \ge 0$ is a real constant such that
\[
	|f(z)| \le M
\]
for all points on $C$, then
\[
	\left| \int_C f(z) dz \right| \le ML
\]
Note that such a number $M$ will always exist because $f$ is continuous on $C$.

\subsection{Examples}
\textit{Example: Suppose that
\[
	f(z) = \frac{z + 4}{z^3 - 1}
\]
and the contour $C$ is a quarter circle with a radius of $2$ in the first quadrant (running from $z = 2$ to $z = 2i$). Since $|z| = 2$ at all points on this contour, we can write that
\[
	|z + 4| \le |z| + 4 = 6
\]
and
\[
	|z^3 - 1| \ge |z|^3 - 1 = 7
\]
Since the length of the contour is $L = \pi$, we can write the upper bound
\[
	\left| \int_C \frac{z + 4}{z^3 - 1} dz \right| \le \frac{6\pi}{7}
\]
}

\textit{Example: suppose that $C_R$ is the semicircular contour
\[
	z = Re^{i\theta}	\quad	(0 \le \theta \le \pi)
\]
and $f$ is the function
\[
	f(z) = \frac{z^{1/2}}{z^2 + 1}
\]
where $z^{1/2}$ denotes the branch $-\pi/2 < \theta < 3\pi/2$. Anywhere on this semicircle,
\[
	|z^{1/2}| = \sqrt{R}
\]
and
\[
	|z^2 + 1| \ge ||z^2| - 1| = R^2 - 1
\]
Since the contour has a length of $\pi R$, the contour integral of $f$ along $C$ can be limited by
\begin{align*}
	\int_C f(z) dz 
	&\le \frac{\sqrt{R}}{R^2 - 1} \cdot \pi R \\
	&= \frac{\pi R^{3/2}}{R^2 - 1} \\
	&= \frac{\pi / \sqrt{R}}{1 - (1 / R^2)}
\end{align*}
As $R$ approaches infinity, this bound approaches zero, so
\[
	\lim_{R \to \infty} \int_C f(z) dz = 0
\]}


\clearpage
\section{Frame 44 -- Antiderivatives}
We saw earlier that some functions have integrals from $z_1$ to $z_2$ that are independent of path. This section will look more closely at such functions to extend the fundamental theorem of calculus.

\subsection{Antiderivatives and their theorem}
Recall that an antiderivative of a continuous function $f(z)$ on a domain $D$ is an analytic function $F(z)$ such that $F'(z) = f(z)$ for all $z$ in $D$. Also, note that antiderivatives are unique to an additive constant: if $F(z)$ and $G(z)$ are two antiderivatives of the same function, then $F'(z) - G'(z) = 0$, so $F(z) - G(z)$ must be constant everywhere.

The following theorem makes several relationships between antiderivatives and their properties.

\textit{Theorem: If $f(z)$ is a continuous function on a domain $D$, then any one of these statements implies the other two:
\begin{enumerate}
	\item $f(z)$ has an antiderivative $F(z)$ throughout $D$;
	\item Any integral of $f(z)$ along a contour in $D$ is path independent, and
	\[
		\int_{z_1}^{z_2} f(z) dz = F(z_2) - F(z_1)
	\]
	\item The integral of $f(z)$ around any closed contour in $D$ is zero. 
\end{enumerate}}

\subsection{Examples}
\textit{Example 1: the function $f(z) = z^2$ has an antiderivative $F(z) = z^3/3$ throughout the entire plane. This allows us to write, for any contour extending from $z = 0$ to $1 + i$,
\begin{align*}
	\int_0^{1+i} z^2 dz 
	&= \frac{z^3}{3} \Big|_0^{1+i} \\
	&= \frac{1}{3}(1 + i)^3 \\
	&= \frac{1}{3} (\sqrt{2} e^{i\pi/4} \\
	&= \frac{2}{3} \sqrt{2} e^{i3\pi/4} \\
	&= \frac{2}{3} (-1 + i)
\end{align*}}


\textit{Example 2: the function $f(z) = \frac{1}{z^2}$ is continuous everywhere except for the point $z = 0$. In this domain, it has an antiderivative $F(z) = \frac{-1}{z}$. Thus, if $C$ is a positively oriented circle centered at the origin,
\[
	\int_C \frac{dz}{z^2} = 0
\]}

\textit{Example 3: the function $f(z) = \frac{1}{z}$ does not have a simple antiderivative on the entire plane. Although $\log z$ is an antiderivative where it is defined, it requires a branch cut to be single valued.}

\textit{Suppose we want to evaluate the integral
\[
	\int_C \frac{dz}{z} 
\]
where $C$ is the full circle
\[
	z = 2e^{i\theta},	\quad -\frac{\pi}{2} \le \theta \le \frac{3\pi}{2}
\]
We can do this by splitting the contour into two legs: $C_1$ is the right semicircle ($-\pi/2 \le \theta \le \pi/2$) and $C_2$ is the left semicircle ($\pi/2 \le \theta \le 3\pi/2$). Then, the principal branch of the logarithm is a suitable antiderivative for $C_1$, so
\[
	\int_{C_1} \frac{dz}{z}
	= \Log(2i) - \Log(-2i) 
	= (\ln 2 + i\pi/2) - (\ln 2 - i\pi/2)
	= i\pi
\]
To evaluate the integral along $C_2$, we switch to the branch $0 < \theta < 2\pi$. Here,
\[
	\int_{C_2} \frac{dz}{z}
	= \log(-2i) - \log(2i)
	= (\ln 2 + i3\pi/2) - (\ln 2 + i\pi/2) 
	= i\pi
\]
so, adding these up,
\[
	\int_C \frac{dz}{z}
	= \int_{C_1} \frac{dz}{z} + \int_{C_2} \frac{dz}{z} 
	= i \pi + i \pi
	= 2\pi i
\]}

\textit{Example 4: suppose we want to evaluate the integral
\[
	\int_{-3}^{3} z^{1/2} dz
\]
where $z^{1/2}$ denotes the branch for $0 < \theta < 2\pi$ and we use any contour that is always above the real axis (except at the endpoints). Since this branch is not defined at the endpoints, we can replace the integrand with the branch
\[
	z^{1/2} = \sqrt{r} e^{i\theta/2}
	\quad (-\frac{\pi}{2} < \theta < \frac{3\pi}{2})
\]
This new function has an antiderivative of $\frac{2}{3} z^{3/2}$, so we can write
\[
	\int_{-3}^{3} z^{1/2} dz
	= R^{3/2} e^{i3\theta/2} \Big|_{-3}^3
	= 2\sqrt{3} (e^0 - e^{i3\pi/2})
	= 2\sqrt{3} (1 + i)
\]	
We could evaluate the same integral for any contour below the real axis in a similar manner.}


\clearpage
\section{Frame 46 -- Cauchy-Goursat Theorem}
\subsection{The theorem}
We will show some simple conditions under which a contour integral is guaranteed to be zero.

Suppose that $C$ is a simple, closed, positively-oriented contour $z(t)$ (where $a \le t \le b$) and $f$ is a function that analytic at each point interior to and on $C$. Then, we can write
\[
	\int_C f(z) dz = \int_a^b f[z(t)] z'(t) dt
\]
Then, if
\[
	f(z) = u(x, y) + iv(x, y) 
\]
and
\[
	z(t) = x(t) + iy(t)
\]
we can write this integral as
\[
	\int_C f(z) dz = \int_C u dx - v dy + i\int_C v dx + u dy
\]
We can then use Green's theorem to write these line integrals as double integrals over the region $R$ bounded by $C$:
\[
	\int_C f(z) dz 
	=  \int \int_R (-v_x - u_y) dA 
	+ i\int \int_R  (u_x - v_y) dA 
\]
Finally, due to the Cauchy Riemann equations, both of these integrands are zero, so
\[
	\int_C f(z) dz = 0
\]
whenever $f$ is analytic and $f'$ is continuous in $R$. Note that, due to Goursat, the continuity condition is actually unnecessary.

\subsection{Example}
\textit{Example: if $C$ is a simple closed contour, then
\[
	\int_C e^{z^3} dz = 0
\]	
since $e^{z^3}$ is analytic everywhere.}



\end{document}