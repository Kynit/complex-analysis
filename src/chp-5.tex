\documentclass{article}

\usepackage{amsmath}
\usepackage{parskip}
\usepackage{tikz}
\renewcommand{\emph}{\textbf}
\renewcommand{\bar}{\overline}
\DeclareMathOperator{\Log}{Log}
\DeclareMathOperator{\Arg}{Arg}
\DeclareMathOperator{\sech}{sech}
\DeclareMathOperator{\csch}{csch}

\begin{document}

\section{Frame 55 -- Sequences and Convergence}
\subsection{Definitions}
An \emph{infinite sequence} of complex numbers,
\[
	z_1, z_2, \dots, z_n, \dots
\]
has a \emph{limit} if, for each positive number $\epsilon$, there exists a positive integer $n_0$ such that
\[
	|z_n - z| < \epsilon	\quad \text{whenever} \quad	n > n_0
\]
Geometrically, this limit implies that for all $n > n_0$, each number $z_n$ in the sequence will be inside an $\epsilon$ neighbourhood of $z$.

A sequence can only have one limit, at most. When this limit exists, we say that the sequence \emph{converges} to $z$, and we write
\[
	\lim_{n \to \infty} z_n = z
\]
If a sequence has no limit, it \emph{diverges}.

\subsection{Components}
\textit{Theorem: If we write $z_n = x_n + iy_n$ and $z = x + iy$, then
\[
	\lim_{n \to \infty} z_n = z \iff
	\lim_{n \to \infty} x_n = x \text{ and }
	\lim_{n \to \infty} y_n = y
\]}
This theorem allows us to write
\[
	\lim_{n \to \infty} (x_n + iy_n)
	= \lim_{n \to \infty} x_n 
	+ i \lim_{n \to \infty} y_n
\]
as long as the limits on either side of this equation exist.

\subsection{Examples}
\textit{Example 1: we can evaluate the following limit easily:
\begin{align*}
	\lim_{n \to \infty} \frac{1}{n^3 + i}
	&= \lim_{n \to \infty} \frac{1}{n^3} + i\lim_{n \to \infty} 1 \\
	&= 0 + i\cdot1 \\
	&= i
\end{align*}
}

\textit{Example 2: Polar coordinates require some extra care. Looking at the sequence
\[
	z_n = -2 + i \frac{(-1)^n}{n^2}
\]
we can see that
\[
	\lim_{n \to \infty} z_n
	= \lim_{n \to \infty} (-2)
	+ i \lim_{n \to \infty} \frac{(-1)^n}{n^2}
	= -2
\]
However, we can find that the principal polar representation of these numbers is
\begin{align*}
	r_n &= \sqrt{4 + \frac{1}{n^2}} \\
	\Theta_n &= \Arg z_n = \tan^{-1} \left( \frac{(-1)^n}{-2n^2}\right)
\end{align*}
Evaluating the first limit, we find that
\[
	\lim_{n \to \infty} r_n = \sqrt{4} = 2
\]
which is fine. However, the second sequence does not converge. Looking at every second term, we see that
\[
	\lim_{n \to \infty} \Theta_{2n} = \pi
\]
and
\[
	\lim_{n \to \infty} \Theta_{2n - 1} = -\pi
\]
so $\Theta_n$ diverges.}


\clearpage
\section{Frame 56 -- Series Convergence}
\subsection{Definitions}
An infinite \emph{series} of complex numbers,
\[
	\sum_{n=1}^\infty z_n = z_1 + z_2 + z_3 + \dots + z_n + \dots,
\]
\emph{converges} to the sum $S$ if the sequence of partial sums,
\[
	S_N = \sum_{n=1}^N z_n = z_1 + z_2 + \dots + z_N,
\]
converges to $S$. If this is the case, then we can write
\[
	\sum_{n=1}^\infty z_n = S
\]

Note that, since a sequence can have at most one limit, a series can have at most one sum. If a series does not converge, it \emph{diverges}.

\subsection{Properties -- Components}
First, as with sequences, we can split a series into its real and imaginary components.

\textit{Theorem: If $z_n = x_n + iy_n$ and $S = X + iY$, then
\[
	\sum_{n=1}^\infty z_n = S
\]
iff
\[
	\sum_{n=1}^\infty x_n = X \quad \text{and} \quad \sum_{n=1}^\infty y_n = Y
\]}

To prove this, we can write the partial sums $S_N$ as
\[
	S_N = X_N + iY_N
\]
where
\[
	X_N = \sum_{n=1}^N x_n \quad \text{and} \quad Y_N = \sum_{n=1}^N y_ns
\]
Then, the series only converges to $S$ if
\[
	\lim_{N \to \infty} X_N = X \quad \text{and} \quad \lim_{N \to \infty} Y_N = Y
\]
due to the theorem on sequences in the previous chapter. Thus, the theorem is proved.

\subsection{Properties -- Boundedness}
The following corollary is a consequence of the previous theorem: 

\textit{Corollary 1: If a series of complex numbers converges, the summed terms $z_n$ converge to zero.}

This is due to the fact that
\[
	\sum_{n=1}^\infty z_n = \sum_{n=1}^\infty x_n + i \sum_{n=1}^\infty y_n
\]
and, in order for these two terms to converge, $x_n$ and $y_n$ must converge to zero (from calculus). Thus,
\[
	\lim_{n \to \infty} z_n 
	= \lim_{n \to \infty} x_n + i \lim_{n \to \infty} y_n
	= 0
\]

This corollary implies that the terms within convergent series are bounded -- that is, there exists a constant $M$ such that $|z_n| < M$ for all $n$.

\subsection{Properties -- Absolute Convergence}
A series is \emph{absolutely convergent} if the related series
\[
	\sum_{n=1}^\infty |z_n| = \sum_{n=1}^\infty \sqrt{x_n^2 + y_n^2}
\]
converges. This has a simple implication:

\textit{Corollary 2: If a series is absolutely convergent, it is convergent.}

To show this, consider the real component of the series. It can be written as
\[
	\sum_{n=1}^\infty x_n
	\le \left| \sum_{n=1}^\infty x_n \right|
	\le \sum_{n=1}^\infty |x_n|
	\le \sum_{n=1}^\infty \sqrt{x_n^2 + y_n^2}
	= 0
\]
so the real component must converge. The same is true of the imaginary component, so the corollary is proved.

\subsection{Remainders}
It is often helpful to define the sequence of \emph{remainders} using the partial sums:
\[
	\rho_N = S - S_N
\]
or $S = S_N + \rho_N$. Since we can write that
\[
	|S_N - S| = |\rho_N|
\]
then a series is only convergent if the sequence of remainders tends to zero.

\textit{Example: using remainders, we can verify that
\[
	\sum_{n=0}^\infty z_n = \frac{1}{1 - z} \quad \text{whenver} \quad |z| < 1
\]
To do this, we recall that
\[
	S_N(z) = 1 + z + z^2 + \dots + z^N = \frac{1 - z^{N+1}}{1 - z}
\]
so
\[
	\rho_N(z) = \frac{1}{1 - z} - \frac{1 - z^{N+1}}{1 - z}
	= \frac{z^N}{1 - z}
\]
The moduli of these remaiders are
\[
	|\rho_N(z)| = \frac{|z|^N}{|1 - z|}
\]
so $\rho_N(z)$ tends to zero when $|z| < 1$.}


\end{document}