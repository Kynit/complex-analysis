\documentclass{article}

\usepackage{amsmath}
\usepackage{parskip}
\usepackage{tikz}
\renewcommand{\emph}{\textbf}
\renewcommand{\bar}{\overline}
\DeclareMathOperator{\Log}{Log}
\DeclareMathOperator{\Arg}{Arg}
\DeclareMathOperator{\sech}{sech}
\DeclareMathOperator{\csch}{csch}

\begin{document}

\section{Frame 55 -- Sequences and Convergence}
\subsection{Definitions}
An \emph{infinite sequence} of complex numbers,
\[
	z_1, z_2, \dots, z_n, \dots
\]
has a \emph{limit} if, for each positive number $\epsilon$, there exists a positive integer $n_0$ such that
\[
	|z_n - z| < \epsilon	\quad \text{whenever} \quad	n > n_0
\]
Geometrically, this limit implies that for all $n > n_0$, each number $z_n$ in the sequence will be inside an $\epsilon$ neighbourhood of $z$.

A sequence can only have one limit, at most. When this limit exists, we say that the sequence \emph{converges} to $z$, and we write
\[
	\lim_{n \to \infty} z_n = z
\]
If a sequence has no limit, it \emph{diverges}.

\subsection{Components}
\textit{Theorem: If we write $z_n = x_n + iy_n$ and $z = x + iy$, then
\[
	\lim_{n \to \infty} z_n = z \iff
	\lim_{n \to \infty} x_n = x \text{ and }
	\lim_{n \to \infty} y_n = y
\]}
This theorem allows us to write
\[
	\lim_{n \to \infty} (x_n + iy_n)
	= \lim_{n \to \infty} x_n 
	+ i \lim_{n \to \infty} y_n
\]
as long as the limits on either side of this equation exist.

\subsection{Examples}
\textit{Example 1: we can evaluate the following limit easily:
\begin{align*}
	\lim_{n \to \infty} \frac{1}{n^3 + i}
	&= \lim_{n \to \infty} \frac{1}{n^3} + i\lim_{n \to \infty} 1 \\
	&= 0 + i\cdot1 \\
	&= i
\end{align*}
}

\textit{Example 2: Polar coordinates require some extra care. Looking at the sequence
\[
	z_n = -2 + i \frac{(-1)^n}{n^2}
\]
we can see that
\[
	\lim_{n \to \infty} z_n
	= \lim_{n \to \infty} (-2)
	+ i \lim_{n \to \infty} \frac{(-1)^n}{n^2}
	= -2
\]
However, we can find that the principal polar representation of these numbers is
\begin{align*}
	r_n &= \sqrt{4 + \frac{1}{n^2}} \\
	\Theta_n &= \Arg z_n = \tan^{-1} \left( \frac{(-1)^n}{-2n^2}\right)
\end{align*}
Evaluating the first limit, we find that
\[
	\lim_{n \to \infty} r_n = \sqrt{4} = 2
\]
which is fine. However, the second sequence does not converge. Looking at every second term, we see that
\[
	\lim_{n \to \infty} \Theta_{2n} = \pi
\]
and
\[
	\lim_{n \to \infty} \Theta_{2n - 1} = -\pi
\]
so $\Theta_n$ diverges.}


\end{document}