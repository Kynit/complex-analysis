\documentclass{article}

\usepackage{amsmath}
\usepackage{parskip}
\usepackage{tikz}
\renewcommand{\emph}{\textbf}
\renewcommand{\bar}{\overline}
\DeclareMathOperator{\Log}{Log}
\DeclareMathOperator{\Arg}{Arg}
\DeclareMathOperator{\sech}{sech}
\DeclareMathOperator{\csch}{csch}
\DeclareMathOperator{\Res}{Res}

\begin{document}

\section{Frame 71 -- Residues}
\textbf{1(a)}
This function is
\begin{align*}
	\frac{1}{z(1 + z)}
	&= \frac{1}{z} \frac{1}{1 + z} \\
	&= \frac{1}{z} (1 - z + z^2 - \dots) \\
	&= \frac{1}{z} - 1 + z - \dots
\end{align*}
so the residue at $0$ is $1$.

\textbf{1(b)}
This function is
\begin{align*}
	z\cos \left(\frac{1}{z} \right)
	&= z \sum_{n=0}^\infty \frac{(-1)^n}{(2n)!} z^{-2n} \\
	&= \sum_{n=0}^\infty \frac{(-1)^n}{(2n)!} z^{1 - 2n} \\
	&= z - \frac{1}{2z} + \frac{1}{24 z^3} - \dots
\end{align*}
so the residue at zero is $-1/2$.

\textbf{1(c)}
This function is
\begin{align*}
	\frac{z - \sin z}{z}
	&= 1 - \frac{\sin z}{z} \\
	&= 1 - \frac{1}{z} \sum_{n=0}^\infty \frac{(-1)^n}{(2n+1)!} z^{2n + 1} \\
	&= 1 - \sum_{n=0}^\infty \frac{(-1)^n}{(2n+1)!} z^{2n} 
\end{align*}
This series has no $1/z$ term, so the residue at zero is $0$.

\textbf{1(d)}
The Laurent series expansion for this function is
\begin{align*}
	\frac{1}{z^4} \cot z
	&= \frac{1}{z^4} \left( \frac{1}{z} - \frac{z}{3} - \frac{z^3}{45} - \frac{2z^5}{945} - \dots \right) \\
	&= \frac{1}{z^5} - \frac{1}{3z^3} - \frac{1}{45z} - \frac{2z}{945} - \dots
\end{align*}
so the residue at $z = 0$ is $-1/45$.

\textbf{1(e)}
A series expansion for this function is
\begin{align*}
	\frac{\sinh z}{z^4 (1 - z^2)}
	&= \frac{1}{z^4} \left( \sum_{n=0}^\infty \frac{1}{(2n+1)!} z^{2n+1} \right) \left(\sum_{n=0}^\infty z^{2n} \right) \\
	&= \frac{1}{z^4} \left( z + \frac{z^3}{6} + \frac{z^5}{120} \right) \left( 1 + z^2 + z^4 + \dots \right) \\
	&= \frac{1}{z^4} \left( z + \frac{7z^3}{6} + \frac{141 z^5}{120} + \dots \right) \\
	&= \frac{1}{z^3} + \frac{7}{6z} + \frac{141z}{120} + \dots
\end{align*}
so the residue at zero is $7/6$.

\textbf{2(a)}
This function only has a singularity at $z = 0$. Finding the Laurent series here, the expansion is
\begin{align*}
	\frac{1}{z^2} e^{-z}
	&= \frac{1}{z^2} \sum_{n=0}^\infty \frac{(-1)^n}{n!} z^n \\
	&= \frac{1}{z^2} \left( 1 - z + \frac{z^2}{2} - \frac{z^3}{6} + \dots \right) \\
	&= \frac{1}{z^2} - \frac{1}{z} + \frac{1}{2} - \frac{z}{6} + \dots 
\end{align*}
so the residue at zero is $-1$, and
\[
	\int_C \frac{e^{-z}}{z^2}~dz = 2\pi i (-1) = -2\pi i
\]

\textbf{2(b)} 
This function now has a singular point at $z = 1$. The series expansion here is
\begin{align*}
	\frac{1}{(z - 1)^2} e^{-z}
	&= \frac{1}{(z - 1)^2} e^{-(z - 1)} \frac{1}{e} \\
	&= \frac{1}{e(z - 1)^2} \sum_{n=0}^\infty \frac{(-1)^n}{n!} (z - 1)^n \\
	&= \frac{1}{e} \left( \frac{1}{(z-1)^2} - \frac{1}{z-1} + \frac{1}{2} - \frac{z-1}{6} + \dots \right)
\end{align*}
so the residue at $z = 1$ is $-1/e$, and
\[
	\int_C f(z)~dz = 2\pi i (-1/e) = -\frac{2\pi}{e} i
\]

\textbf{2(c)}
This function only has a singular point at $z = 0$, with the series expansion
\begin{align*}
	z^2 e^{1/z}
	&= z^2 \sum_{n=0}^\infty \frac{1}{n!} z^{-n} \\
	&= z^2 \left(1 + \frac{1}{z} + \frac{1}{2z^2} + \frac{1}{6z^3} + \dots \right) \\
	&= z^2 + z + \frac{1}{2} + \frac{1}{6z} + \dots 
\end{align*}
so the residue here is $1/6$, and
\[
	\int_C z^2 e^{1/z} dz = 2\pi i \frac{1}{6} = \frac{\pi i}{3}
\]

\textbf{2(d)}
This function has singular points at $z = 0$ and $z = 2$. Expanding the function at $z = 0$ gives
\begin{align*}
	\frac{z + 1}{z} \frac{1}{z - 2}
	&= \left(1 + \frac{1}{z} \right) \frac{-1}{2(1 - z/2)} \\
	&= \left(1 + \frac{1}{z} \right) \frac{-1}{2} \left(1 + \frac{z}{2} + \frac{z^2}{4} + \frac{z^3}{8} + \dots\right) \\
	&= -\frac{1}{2z} - \frac{3}{2} - \frac{3z}{4} - \dots
\end{align*}
so the residue at $z = 0$ is $-1/2$. Then,
\begin{align*}
	\frac{z + 1}{z - 2} \frac{1}{z}
	&= \frac{(z - 2) + 3}{z - 2} \frac{1}{2 + (z - 2)} \\
	&= \frac{1}{2} \left(1 + \frac{3}{z - 2} \right) \frac{1}{1 + (z-2)/2} \\
	&= \frac{1}{2} \left(1 + \frac{3}{z - 2} \right) \left( 1 - \frac{z-2}{2} + \frac{(z-2)^2}{4} - \dots \right) \\
	&= \frac{3}{2 (z-2)} - \frac{1}{4} + \dots
\end{align*}
so the residue at $z = 2$ is $3/2$. Thus,
\[
	\int_C \frac{z + 1}{z^2 - 2z} dz = 2\pi i (-1/2 + 3/2) = 2\pi i
\]

\textbf{3(a)}
The residue at infinity can be found by writing the function
\begin{align*}
	\frac{1}{z^2} \frac{(1 / z)^5}{1 - (1/z)^3}
	&= \frac{-1}{z^4} \frac{1}{1 - z^3} \\
	&= \frac{-1}{z^4} \left(1 + z^3 + z^6 + \dots \right) \\
	&= -\frac{1}{z^4} - \frac{1}{z} - z^2 - \dots
\end{align*}
so the residue at infinity is $- (-1)$, and
\[
	\int_C f(z)~dz = 2\pi i \cdot (-1) = -2\pi i
\] 

\textbf{3(b)}
The residue at infinity can be found via
\begin{align*}
	\frac{1}{z^2} \frac{1}{1 + (1/z)^2} 
	&= \frac{1}{1 + z^2} \\
	&= 1 - z^2 + z^4 - \dots
\end{align*}
so the residue at infinity is zero, and
\[
	\int_C f(z)~dz = 0
\]

\textbf{3(c)}
The residue at infinity, from 
\[
	\frac{1}{z^2} \frac{1}{1/z}
	= \frac{1}{z} 
\]
is $-1$, so
\[
	\int_C f(z)~dz = 2\pi i
\]


\clearpage
\section{Frame 72 -- Singular Points}
\textbf{1(a)}
This function is
\[
	ze^{1/z}
	= z\left(1 + \frac{1}{z} + \frac{1}{2z^2} + \dots \right)
	= z + 1 + \frac{1}{2z} + \dots
\]
so it has an essential singular point at the origin.

\textbf{1(b)}
This function is
\[
	\frac{z^2}{z + 1}
	= \frac{(z+1)^2 - 2(z+1) + 1}{z + 1}
	= (z + 1) - 2 + \frac{1}{z + 1}
\]
so it has a simple pole at $z = -1$.

\textbf{1(c)}
This function is
\[
	\frac{\sin z}{z}
	= \frac{1}{z} \left(z - \frac{z^3}{3!} + \frac{z^5}{5!} - \dots \right)
	= 1 - \frac{z^2}{3!} + \frac{z^4}{5!} - \dots
\]
so it has a removable singular point at the origin.

\textbf{1(d)}
This function is
\[
	\frac{\cos z}{z}
	= \frac{1}{z} \left(1 - \frac{z^2}{2!} + \frac{z^4}{4!} - \dots \right)
	= \frac{1}{z} - \frac{z}{2!} + \frac{z^3}{4!} - \dots
\]
so it has a simple pole at the origin.

\textbf{1(e)}
This function is already in principal form. It has a third order pole at $z = 2$.

\textbf{2(a)}
This function is
\begin{align*}
	\frac{1 - \cosh z}{z^3}
	&= \frac{1}{z^3} \left[ 1 - \left( 1 + \frac{z^2}{2!} + \frac{z^4}{4!} + \dots \right) \right] \\
	&= \frac{1}{z^3} \left[ -\frac{z^2}{2!} - \frac{z^4}{4!} - \dots \right] \\
	&= -\frac{1}{2! \cdot z} - \frac{z}{4!} - \dots
\end{align*}
so it has a first-order pole at the origin with a residue of $B = -1/2$.

\textbf{2(b)}
This function is
\begin{align*}
	\frac{1 - e^{2z}}{z^4}
	&= \frac{1}{z^4} \left[ -\frac{2z}{1!} - \frac{4z^2}{2!} - \frac{8z^3}{3!} - \frac{16z^4}{4!} \dots \right] \\
	&= -\frac{2}{z^3} - \frac{2}{z^2} - \frac{4}{3z} - \frac{2}{3} - \dots
\end{align*}
so it has a third-order pole at the origin with a residue of $B = -4/3$.

\textbf{4}
To solve the equation
\[
	e^{1/z} = -1
\]
we note that this occurs when
\[
	\frac{1}{z} = (2n + 1) \pi i
\]
or
\[
	z = \frac{1}{(2n + 1)\pi i} = -\frac{i}{(2n + 1)\pi}
\]

\textbf{5}
If we write the function
\[
	f(z) = \frac{8a^3z^2}{(z^2 + a^2)^3}
\]
as
\[
	f(z) = \frac{\phi(z)}{(z - ai)^3} \quad \text{where} \quad
	\phi(z) = \frac{8a^3z^2}{(z + ai)^3}
\]
then, since $\phi(z)$ has no singular points at $z = ai$, we can write its Taylor series as
\begin{align*}
	\phi(z)
	&= \frac{8a^3z^2}{(z + ai)^3} \\
	&= \phi(ai) + \phi'(ai) (z - ai) + \frac{\phi''(ai)}{2} (z - ai)^2 + \dots
\end{align*}
To find these coefficients, the derivative of $\phi(z)$ is
\begin{align*}
	\phi'(z)
	&= \frac{d}{dz} \frac{8a^3z^2}{(z + ai)^3} \\
	&= \frac{16a^3z (z + ai)^3 - 24a^3z^2(z + ai)^2}{(z + ai)^6} \\
	&= \frac{16a^3z (z + ai) - 24a^3z^2}{(z + ai)^4} \\
	&= \frac{8a^3z (-z + 2ai)}{(z + ai)^4}
\end{align*}
and the second derivative is
\begin{align*}
	\phi''(z)
	&= \frac{d}{dz} \frac{8a^3z (-z + 2ai)}{(z + ai)^4} \\
	&= \frac{d}{dz} \frac{-8a^3z^2 + 16a^4z i}{(z + ai)^4} \\
	&= \frac{(-16a^3z + 16a^4 i)(z + ai)^4 - 4(z + ai)^3(-8a^3z^2 + 16a^4z i)}{(z + ai)^8} \\
	&= \frac{(-16a^3z + 16a^4 i)(z + ai) - 4(-8a^3z^2 + 16a^4z i)}{(z + ai)^5}
\end{align*}
Evaluating these at $z = ai$,
\begin{align*}
	\phi(ai) 
	&= \frac{8a^3(ai)^2}{(2ai)^3} \\
	&= - i \frac{8a^5}{8a^3} \\
	&= - a^2i
\end{align*}
\begin{align*}
	\phi'(ai)
	&= \frac{8a^3(ai)^2}{(2ai)^4} \\
	&= -\frac{8a^5}{16a^4} \\
	&= -\frac{a}{2}
\end{align*}
\begin{align*}
	\phi''(ai)
	&= \frac{(-16a^3(ai) + 16a^4 i)(2ai) - 4(8a^3(ai)^2)}{(2ai)^5} \\
	&= \frac{(0) + 4(8a^5)}{32a^5 i} \\
	&= -i
\end{align*}
so we find that
\[
	\phi(z)
	= -a^2 i - \frac{a}{2} (z - ai) - \frac{i}{2} (z - ai)^2
\]
and
\[
	f(z) = \frac{\phi(z)}{(z - ai)^3}
	= \frac{-i/2}{z - ai} - \frac{a/2}{(z - ai)^2} - \frac{a^2 i}{(z - ai)^3}
\]


\clearpage
\section{Frame 74 -- Residues and Poles}
\textbf{1(a)}
This function has a simple pole at $z = 1$, with
\[
	B = \phi(1) = (1)^2 + 2 = 3 
\]

\textbf{1(b)}
This function has a third-order pole at $z = -1/2$, with
\[
	B = \frac{(-1/2)^3}{2^3} = -1/64
\]

\textbf{1(c)}
This function has simple poles at $z = \pm \pi i$. At $z = \pi$,
\[
	B_1 = \frac{e^z}{z + i\pi} \Big|_{z = i\pi}
	= \frac{e^{i\pi}}{2i\pi}
	= \frac{i}{2\pi}
\]
and at $z = -\pi$,
\[
	B_2 = \frac{e^z}{z - i\pi} \Big|_{z = -i\pi}
	= \frac{e^{-i\pi}}{-2i\pi}
	= \frac{-i}{2\pi}
\]

\textbf{2(a)}
This residue is
\[
	\Res_{z = -1} \frac{z^{1/4}}{z + 1}
	= z^{1/4} \Big|_{z = -1}
	= 1e^{\pi/4}
	= \frac{1 + i}{\sqrt{2}}
\]

\textbf{2(b)}
This residue is
\begin{align*}
	\Res_{z = i} \frac{\Log z}{(z^2 + 1)^2}
	&= \frac{d}{dz} \frac{\Log z}{(z + i)^2} \Big|_{z = i} \\
	&= \frac{\frac{1}{z}(z + i)^2 - 2(z + i)\Log z}{(z + i)^4} \Big|_{z = i} \\
	&= \frac{1 + i/z - 2\Log z}{(z + i)^3} \Big|_{z = i} \\
	&= \frac{1 + 1 - 2(0 + i\pi/2)}{-8i} \\
	&= \frac{2 - i \pi}{-8i} \\
	&= \frac{\pi + 2i}{8}
\end{align*}

\textbf{3(a)}
This circle only contains the singular point $z = 1$, so
\[
	\int_C f(z)~dz
	= 2\pi i \cdot \Res_{z = 1} f(z)
	= 2\pi i \cdot \frac{3z^3 + 2}{z^2 + 9} \Big|_{z = 1}
	= 2\pi i \cdot \frac{3 + 2}{1 + 9}
	= \pi i
\]

\textbf{3(b)}
Now, the circle contains every singular point of $f$. The residue at infinity is then
\begin{align*}
	\Res_{z = \infty} f(z)
	&= -\Res_{z = 0} \frac{1}{z^2} f(1/z) \\
	&= -\Res_{z = 0} \frac{1}{z^2} \frac{3 + 2z^3}{(1 - z)(1 + 9z^2)} \\
	&= -\frac{d}{dz} \frac{3 + 2z^3}{-9z^3 + 9z^2 - z + 1} \Big|_{z = 0} \\
	&= -\frac{(6z^2)(-9z^3 + 9z^2 - z + 1) - (-27z^2 + 18z - 1)(3 + 2z^3)}{(1 - z + 9z^2 - 9z^3)^2} \Big|_{z = 0} \\
	&= -\frac{0 - (-1)(3)}{1} \\
	&= -3
\end{align*}
so
\[
	\int_C f(z)~dz = -2\pi i(-3) = 6\pi i
\]

\textbf{4(a)}
This contour only contains the singular point at $z = 0$. The residue here is
\begin{align*}
	\Res_{z = 0} f(z)
	&= \frac{1}{2} \frac{d^2}{dz^2} \frac{1}{z + 4} \Big|_{z = 0} \\
	&= \frac{1}{2} \frac{d}{dz} \frac{-1}{(z + 4)^2} \Big|_{z = 0} \\
	&= \frac{1}{2} \frac{2}{(z + 4)^3} \Big|_{z = 0} \\
	&= \frac{1}{64}
\end{align*}
so
\[
	\int_C f(z)~dz = 2\pi i \frac{1}{64} = \frac{\pi i}{32}
\]

\textbf{4(b)}
This contour now contains both singular points. The second residue (at $z = -4$) is
\[
	\Res_{z = -4} f(z)
	= \frac{1}{z^3} \Big|_{z = -4}
	= -\frac{1}{64}
\]
so the sum of the two residues is zero, and
\[
	\int_C f(z)~dz = 0
\]


\clearpage
\section{Frame 76 -- Poles and Zeroes}
\textbf{1}
The residue at $z = 0$ is
\[
	B = \frac{1}{\cos 0} = 1
\]

\textbf{2(a)}
The derivative of the denominator is
\[
	\frac{d}{dz} z^2 \sinh z
	= z^2 \cosh z + 2z \sinh z
\]
At $z = i\pi$, then, the residue is
\[
	B
	= \frac{i \pi - \sinh(i \pi)}{(i\pi)^2 \cosh(i\pi) + (i\pi) \sinh(i\pi)}
	= \frac{i \pi}{ -\pi^2 \cdot (-1)}
	= \frac{i}{\pi}
\]

\textbf{3(a)}
Since this function is $\frac{z}{\cos z}$, the residue is
\[
	B 
	= \frac{z_n}{-\sin(z_n)}
	= \frac{z_n}{-(-1)^n}
	= (-1)^{n+1} z_n
\]

\textbf{4(a)}
Since $\tan z = \sin z / \cos z$, this contour contains the two poles at $z = \pm \pi/2$. Here, the residues are
\[
	B_n
	= \frac{\sin(\pm \pi/2)}{-\sin(\pm \pi/2)}
	= -1
\]
so the sum of the residues is $-2$ and the contour integral evaluates to
\[
	\int_C \tan z~dz = 2\pi i(-2) = -4\pi i
\]

\end{document}