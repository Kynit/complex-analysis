\documentclass{article}

\usepackage{amsmath}
\usepackage{parskip}
\usepackage{tikz}
\renewcommand{\emph}{\textbf}
\renewcommand{\bar}{\overline}
\DeclareMathOperator{\Log}{Log}
\DeclareMathOperator{\Arg}{Arg}
\DeclareMathOperator{\sech}{sech}
\DeclareMathOperator{\csch}{csch}
\DeclareMathOperator{\Res}{Res}

\begin{document}

\section{Frame 78 -- Evaluating Improper Integrals}
\subsection{Improper Integrals}
The \emph{improper integral} of a continuous function $f(x)$ over the interval $0 \le x < \infty$ is defined as
\[
	\int_0^\infty f(x)~dx = \lim_{R \to \infty} \int_0^R f(x)~dx
\]
If this limit exists, we say that the improper integral \emph{converges} to this limit. The improper integral of $f$ over the infinite interval $-\infty < x < \infty$ is
\[
	\int_{-\infty}^\infty 
	= \lim_{R_1 \to \infty} \int_{-R_1}^0 f(x)~dx
	+ \lim_{R_2 \to \infty} \int_0^{R_2}  f(x)~dx
\]
If both of these limits exist, we say that the integral converges to their sum.

\subsection{The Cauchy Principal Value}
We say that the \emph{Cauchy Principal Value} of an indefinite integral is
\[
	\text{P.V.} \int_{-\infty}^\infty f(x)~dx
	= \lim_{R \to \infty} \int_{-R}^R f(x)~dx
\]
as long as this limit exists. 

If the regular improper integral converges, then
\begin{align*}
	\lim_{R \to \infty} \int_{-R}^R f(x)~dx 
	&= \lim_{R \to \infty} \left[ \int_{-R}^0 f(x)~dx + \int_0^R f(x)~dx \right] \\
	&= \lim_{R \to \infty} \int_{-R}^0 f(x)~dx + \lim_{R \to \infty} \int_0^R f(x)~dx
\end{align*}
so the principal value also exists. However, the converse is not true -- the existence of the principal value does not imply the existence of the improper integral.

Next, suppose that $f(x)$ is an even function; ie:
\[
	f(-x) = f(x)
\]
and assume that the principal value exists. Then, the evenness of $f$ allows us to write
\begin{align*}
	\int_{-R_1}^0 f(x)~dx &= \frac{1}{2} \int_{-R_1}^{R_1} f(x)~dx \\
	\int_0^{R_2}  f(x)~dx &= \frac{1}{2} \int_{-R_2}^{R_2} f(x)~dx
\end{align*}
so we can convert both single-sided limits into double-sided limits, and
\[
	\int_{-\infty}^\infty f(x)~dx = \text{P.V.} \int_{-\infty}^\infty f(x)~dx
\]
Also, extending this formula,
\[
	\int_0^\infty f(x)~dx = \frac{1}{2} \text{P.V.} \int_{-\infty}^\infty f(x)~dx
\]

\subsection{Using Residues}
Now, we apply our knowledge of residues to integrate $f(z) = p(z)/q(z)$ along the real axis when $p$ and $q$ are polynomials. In this discussion, suppose that $q$ has at least one zero above the real axis and no zeroes on the real axis.

From our knowledge of polynomials, we know that $q$ has a finite number of distinct zeroes, which we can label as $z_1$, $z_2$, $\dots$, $z_n$. Then, we can integrate the function $f(z)$ along the contour:
\begin{itemize}
	\item Along the real axis from $-R$ to $R$;
	\item Along the upper semicircle with a radius of $R$ from $(R, 0$ to $(-R, 0)$, which we call $C_R$.
\end{itemize}
This contour allows us to write
\[
	\int_{-R}^R f(x)~dx + \int_{C_R} f(z)~dz 
	= 2\pi i \sum_{k=1}^n \Res_{z = z_k} f(z)
\]
or
\[
	\int_{-R}^R f(x)~dx  
	= 2\pi i \sum_{k=1}^n \Res_{z = z_k} f(z)
	- \int_{C_R} f(z)~dz
\]

Using this expression, we can say that if
\[
	\lim_{R \to \infty} \int_{C_R} f(z)~dz = 0
\]
then the following three equations hold (the latter two if $f$ is even):
\begin{align*}
	\text{P.V.}\int_{-\infty}^\infty f(x)~dx &= 2\pi i\sum{k=1}^n \Res_{z=z_k} f(z) \\
	\int_{-\infty}^\infty f(x)~dx &= 2\pi i \sum_{z=z_k} f(z) \\
	\int_{0}^\infty       f(x)~dx &=  \pi i \sum_{z=z_k} f(z)
\end{align*}


\clearpage
\section{Frame 79 -- Example}
This section will show a sample integral that can be calculated using the method from the previous section.

The goal of this example is to calculate
\[
	\int_{0}^\infty \frac{x^2}{x^6 + 1}
\]
To do this, we can define 
\[
	f(z) = \frac{z^2}{z^6 + 1}
\]
and note that this has isolated singularities at the sixth roots of $-1$, or
\[
	c_k = e^{i (1 + 2k) \pi/6}
\]
Three of these roots lie on the upper half-plane, at
\begin{align*}
	c_0 &= e^{i\pi/6} \\
	c_1 &= e^{i\pi/2} = i \\
	c_2 &= e^{i5\pi/6}
\end{align*}
We can find the residue at these three points through the formula
\[
	B_k 
	= \Res_{z=c_k} \frac{z^2}{z^6 + 1}
	= \frac{z^2}{6z^5} \Big|_{z=c_k}
	= \frac{1}{6z^3} \Big|_{z=c_k}
	= \frac{1}{6c_k^3}
\]
so we can write
\[
	2\pi i \sum_{k=1}^n B_k
	= 2\pi i \left( \frac{1}{6i} - \frac{1}{6i} + \frac{1}{6i} \right)
	= \frac{\pi}{3}
\]
and, as long as $R > 1$,
\[
	\int_{-R}^R f(x)~dx = \frac{\pi}{3} - \int_{C_R} f(z)~dz
\]

Next, when $|z| = R$, we can write
\[
	\frac{|z^2|}{|z^6 + 1|} < \frac{R^2}{R^6 - 1}
\]
so 
\[
	\left| \int_{C_R} f(z)~dz \right| 
	\le \frac{R^2}{R^6 - 1} \pi R
	= \frac{R^3}{R^6 - 1}
\]
and, as $R \to \infty$, this approaches $0$. Thus,
\[
	\text{P.V.} \int_{-\infty}^\infty f(z)~dz
	= \frac{\pi}{3}
\]
so
\[
	\int_0^\infty \frac{x^2}{x^6 + 1} = \frac{\pi}{6}
\]


\clearpage
\section{Frame 80 -- Improper Integrals from Fourier Analysis}
In this section, we will look at integrals of the form
\[
	\int_{-\infty}^\infty f(x) \sin ax~dx
\]
and
\[
	\int_{-\infty}^\infty f(x) \cos ax~dx
\]
where $f(x) = p(x)/q(x)$ is a rational function with no poles on the real axis, but at least one above it.

Note that we can't apply the process from the previous section directly, since
\[
	|\sin az|^2 = \sin^2 ax + \sinh^2 ay
\]
is unbounded as $y \to \infty$.

\subsection{Example}
Consider the integral
\[
	\int_{-\infty}^\infty \frac{\cos 3x}{(x^2 + 1)^2} dx = \frac{2\pi}{e^3}
\]
We can begin by defining the function
\[
	f(z) = \frac{1}{(z^2 + 1)^2}
\]
and using the regular semicircular contour from $-R$ to $+R$. Since $f(z)$ only has a single pole in the upper half-plane (at $z = i$), as long as $R > 1$, we can write
\[
	\int_{-R}^R \frac{e^{i3x}}{(x^2 + 1)^2}dx 
	= 2\pi i \Res_{z=i} [f(z) e^{i3z}]
	- \int_{C_R} f(z) e^{i3z} dz
\]
Now, this pole at $z=i$ is a second-order pole, so
\begin{align*}
	B
	&= \frac{d}{dz} \frac{e^{i3z}}{(z + i)^2} \Big|_{z=i} \\
	&= \frac{3ie^{3iz} (z+i)^2 - 2e^{3iz}(z+i)}{(z+i)^4} \Big|_{z=i} \\
	&= \frac{e^{3iz} [3iz - 5]}{(z + i)^3} \Big|_{z=i} \\
	&= \frac{-8e^{-3}}{(2i)^3} \\
	&= \frac{1}{ie^3}
\end{align*}

Next, the integral on the semicircle satisfies
\begin{align*}
	\left| \int_{C_R} f(z) e^{i3z} \right|
	&\le \pi R \cdot \frac{|e^{-3y}|}{(R^2 - 1)^2} \\
	&\le \pi R \cdot \frac{1}{(R^2 - 1)^2}
\end{align*}
and this vanishes as $R$ tends to infinity. Thus, equating real parts of the integral formula,
\[
	\lim_{R \to \infty} \int_{-R}^R \frac{\cos 3x}{(x^2 + 1)^2}
	= 2\pi i \frac{1}{ie^3}
	= \frac{2\pi}{e^3}
\]


\clearpage
\section{Frame 81 -- Jordan's Lemma}
\subsection{Jordan's Lemma}
The following lemma is helpful for evaluating the integrals from the previous section.

\textit{Theorem: Suppose that:
\begin{itemize}
	\item A function $f$ is analytic at all points above the real axis outside some circle $|z| = R_0$
	\item $C_R$ is a semicircle $z = Re^{i\theta}$ ($0 \le \theta \le \pi$) where $R > R_0$
	\item On $C_R$, there is a constant $M_R$ such that
	\[
		|f(z)| \le M_R
	\]
	and
	\[
		\lim_{R \to \infty} M_R = 0
	\]
\end{itemize}
Then, for any positive $a$,
\[
	\lim_{R \to \infty} \int_{C_R} f(z) e^{iaz} dz = 0
\]}

To prove this, we can write
\[
	\int_{C_R} f(z) e^{iaz} dz
	= \int_0^\pi f(Re^{i\theta}) \exp(iaRe^{i\theta}) Rie^{i\theta} d\theta
\]
Then, we know that
\[
	|f(Re^{i\theta})| \le M_R
\]
and
\[
	|\exp(iaRe^{i\theta})| \le e^{-aR \sin \theta}
\]
Next, Jordan's inequality tells us that
\[
	\int_0^\pi e^{-R\sin \theta} d\theta < \frac{\pi}{R}
\]
so we can write the contour integral as
\begin{align*}
	\left| \int_{C_R} f(z) e^{iaz} dz \right|
	&\le M_R R \int_0^\pi e^{-aR\sin \theta} d\theta \\
	&< M_R R \frac{\pi}{aR} \\
	&= \frac{M_R \pi}{a}
\end{align*}
and this tends to zero as $R \to \infty$.

\subsection{Example}
We can use this theorem to help us find the principal value of the integral
\[
	\int_{-\infty}^\infty \frac{x \sin x}{x^2 + 2x + 2} dx
\]
If we write
\[
	f(z) = \frac{z}{z^2 + 2z + 2}
\]
then we can find the residue at the point $z_1 = -1 + i$ as
\[
	B = \frac{z_1 e^{iz_1}}{z_1 - \bar{z_1}}
\]
Next, we can use the standard contour integral to write
\[
	\int_{-R}^R \frac{xe^{ix}}{x^2 + 2x + 2}
	= 2\pi i B
	- \int_{C_R} f(z) e^{iz}~dz
\]
and, taking the imaginary components of both sides,
\[
	\int_{-R}^R \frac{x\sin x}{x^2 + 2x + 2}~dx
	= \Im(2\pi i B)
	- \Im \int_{C_R} f(z) e^{iz}~dz
\]
Then, if $|z| = R$, we can write
\[
	|f(z)| \le \frac{R}{(R - \sqrt{2})^2}
\]
and this tends to zero as $R \to \infty$, so the integral vanishes, leaving
\[
	\text{P.V.} \int_{-\infty}^\infty \frac{x\sin x}{x^2 + 2x + 2}~dx
	= \Im(2\pi i B)
	= \frac{\pi}{e} (\sin 1 + \cos 1)
\]


\clearpage
\section{Frame 82 -- Indented Paths}
\subsection{Main Theorem}
This section will examine the use of indented paths to carry out some more integrals. The following theorem will be useful here:

\textit{Theorem: Suppose that $f$ is a function with a simple pole at $z = x_0$ on the real axis with a Laurent series that is valid for $0 < |z - x_0| < R_2$ and with a residue of $B_0$. Then, if $C_\rho$ is the clockwise upper semicircle
\[
	|z - x_0| = \rho	\quad	(\rho < R_2)
\]
then
\[
	\lim_{\rho \to 0} \int_{C_\rho} f(z)~dz = -B_0 \pi i
\]}

To prove this, we can write $f$ as
\[
	f(z) = g(z) + \frac{B_0}{z - x_0}
\]
where
\[
	g(z) = \sum_{n=0}^\infty a_n (z - x_0)^n
\]
Then, we can write the integral as
\[
	\int_{C_\rho} f(z) = \int_{C_\rho} g(z) + B_0 \int_{C_\rho} \frac{1}{z - x_0}
\]
and look at these two terms separately. First, $g(z)$ is continuous, so it must be bounded on some disk centered at $\rho_0$ (so that $|g(z)| \le M$). This allows us to write
\[
	\left| \int_{C_\rho} g(z)~dz \right|
	\le ML
	= M \pi \rho
\]
which vanishes as $\rho \to 0$. For the second term, we can write out the contour as the equation
\[
	z = x_0 + \rho e^{i\theta}
\]
so that
\[
	\int_{C_\rho} \frac{dz}{z - x_0}
	= -\int_0^\pi \frac{1}{\rho e^{i\theta}} \rho i e^{i\theta}~d\theta
	= -i \int_0^\pi d\theta
	= -i\pi
\]
so
\[
	\int_{C_\rho} f(z)~dz = -B_0 \pi i
\]

\subsection{Example}
We can evaluate the integral
\[
	\int_0^\infty \frac{\sin x}{x}~dx
\]
by using the following contour:
\begin{itemize}
	\item $L_1$, the real axis from $\rho$ to $R$.
	\item $C_R$, a positively-oriented semicircle from $R$ to $-R$;
	\item $L_2$, the real axis from $-R$ to $-\rho$;
	\item $C_\rho$, a negatively-oriented semicircle from $-\rho$ to $\rho$;
\end{itemize}
We do this to avoid the singularity at $z = 0$ in $e^{iz} / z$. Using the Cauchy-Goursat theorem, we can write
\[
	\int_{C_R}      \frac{e^{iz}}{z}~dz
	+ \int_{L_1}    \frac{e^{iz}}{z}~dz
	+ \int_{C_\rho} \frac{e^{iz}}{z}~dz
	+ \int_{L_2}    \frac{e^{iz}}{z}~dz
\]
Now, since we can write $L_1$ as
\[
	z = re^{i0}	\quad (\rho \le r \le R)
\]
and $-L_2$ as
\[
	z = re^{i\pi} = -r \quad (\rho \le r \le R)
\]
we can combine the integrals along the line segments into
\[
	\int_\rho^R \frac{e^{ir}}{r}~dr - \int_\rho^R \frac{e^{-ir}}{r}~dr
	= 2i \int_{\rho}^R \frac{\sin r}{r}~dr
\]
ie:
\[
	2i \int_\rho^R \frac{\sin r}{r}~dr 
	= - \int_{C_\rho} \frac{e^{iz}}{z}~dz
	  - \int_{C_R}    \frac{e^{iz}}{z}~dz
\]
Now, according to the theorem above, since this function has a residue of $1$ at the origin, we can write
\[
	\lim_{\rho \to 0} \int_{C_\rho} \frac{e^{iz}}{z}~dz = -\pi i
\]
and, since $|1 / z| = 1 / R$, we know from Jordan's lemma that
\[
	\lim_{R \to \infty} \int_{C_R} \frac{e^{iz}}{z}~dz = 0
\]
so, applying these limits, we find that
\[
	\int_0^\infty \frac{\sin r}{r}~dr = \frac{1}{2i} -(-\pi i) = \frac{\pi}{2}
\]


\end{document}