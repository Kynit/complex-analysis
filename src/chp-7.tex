\documentclass{article}

\usepackage{amsmath}
\usepackage{parskip}
\usepackage{tikz}
\renewcommand{\emph}{\textbf}
\renewcommand{\bar}{\overline}
\DeclareMathOperator{\Log}{Log}
\DeclareMathOperator{\Arg}{Arg}
\DeclareMathOperator{\sech}{sech}
\DeclareMathOperator{\csch}{csch}
\DeclareMathOperator{\Res}{Res}

\begin{document}

\section{Frame 78 -- Evaluating Improper Integrals}
\subsection{Improper Integrals}
The \emph{improper integral} of a continuous function $f(x)$ over the interval $0 \le x < \infty$ is defined as
\[
	\int_0^\infty f(x)~dx = \lim_{R \to \infty} \int_0^R f(x)~dx
\]
If this limit exists, we say that the improper integral \emph{converges} to this limit. The improper integral of $f$ over the infinite interval $-\infty < x < \infty$ is
\[
	\int_{-\infty}^\infty 
	= \lim_{R_1 \to \infty} \int_{-R_1}^0 f(x)~dx
	+ \lim_{R_2 \to \infty} \int_0^{R_2}  f(x)~dx
\]
If both of these limits exist, we say that the integral converges to their sum.

\subsection{The Cauchy Principal Value}
We say that the \emph{Cauchy Principal Value} of an indefinite integral is
\[
	\text{P.V.} \int_{-\infty}^\infty f(x)~dx
	= \lim_{R \to \infty} \int_{-R}^R f(x)~dx
\]
as long as this limit exists. 

If the regular improper integral converges, then
\begin{align*}
	\lim_{R \to \infty} \int_{-R}^R f(x)~dx 
	&= \lim_{R \to \infty} \left[ \int_{-R}^0 f(x)~dx + \int_0^R f(x)~dx \right] \\
	&= \lim_{R \to \infty} \int_{-R}^0 f(x)~dx + \lim_{R \to \infty} \int_0^R f(x)~dx
\end{align*}
so the principal value also exists. However, the converse is not true -- the existence of the principal value does not imply the existence of the improper integral.

Next, suppose that $f(x)$ is an even function; ie:
\[
	f(-x) = f(x)
\]
and assume that the principal value exists. Then, the evenness of $f$ allows us to write
\begin{align*}
	\int_{-R_1}^0 f(x)~dx &= \frac{1}{2} \int_{-R_1}^{R_1} f(x)~dx \\
	\int_0^{R_2}  f(x)~dx &= \frac{1}{2} \int_{-R_2}^{R_2} f(x)~dx
\end{align*}
so we can convert both single-sided limits into double-sided limits, and
\[
	\int_{-\infty}^\infty f(x)~dx = \text{P.V.} \int_{-\infty}^\infty f(x)~dx
\]
Also, extending this formula,
\[
	\int_0^\infty f(x)~dx = \frac{1}{2} \text{P.V.} \int_{-\infty}^\infty f(x)~dx
\]

\subsection{Using Residues}
Now, we apply our knowledge of residues to integrate $f(z) = p(z)/q(z)$ along the real axis when $p$ and $q$ are polynomials. In this discussion, suppose that $q$ has at least one zero above the real axis and no zeroes on the real axis.

From our knowledge of polynomials, we know that $q$ has a finite number of distinct zeroes, which we can label as $z_1$, $z_2$, $\dots$, $z_n$. Then, we can integrate the function $f(z)$ along the contour:
\begin{itemize}
	\item Along the real axis from $-R$ to $R$;
	\item Along the upper semicircle with a radius of $R$ from $(R, 0$ to $(-R, 0)$, which we call $C_R$.
\end{itemize}
This contour allows us to write
\[
	\int_{-R}^R f(x)~dx + \int_{C_R} f(z)~dz 
	= 2\pi i \sum_{k=1}^n \Res_{z = z_k} f(z)
\]
or
\[
	\int_{-R}^R f(x)~dx  
	= 2\pi i \sum_{k=1}^n \Res_{z = z_k} f(z)
	- \int_{C_R} f(z)~dz
\]

Using this expression, we can say that if
\[
	\lim_{R \to \infty} \int_{C_R} f(z)~dz = 0
\]
then the following three equations hold (the latter two if $f$ is even):
\begin{align*}
	\text{P.V.}\int_{-\infty}^\infty f(x)~dx &= 2\pi i\sum{k=1}^n \Res_{z=z_k} f(z) \\
	\int_{-\infty}^\infty f(x)~dx &= 2\pi i \sum_{z=z_k} f(z) \\
	\int_{0}^\infty       f(x)~dx &=  \pi i \sum_{z=z_k} f(z)
\end{align*}


\clearpage
\section{Frame 79 -- Example}
This section will show a sample integral that can be calculated using the method from the previous section.

The goal of this example is to calculate
\[
	\int_{0}^\infty \frac{x^2}{x^6 + 1}
\]
To do this, we can define 
\[
	f(z) = \frac{z^2}{z^6 + 1}
\]
and note that this has isolated singularities at the sixth roots of $-1$, or
\[
	c_k = e^{i (1 + 2k) \pi/6}
\]
Three of these roots lie on the upper half-plane, at
\begin{align*}
	c_0 &= e^{i\pi/6} \\
	c_1 &= e^{i\pi/2} = i \\
	c_2 &= e^{i5\pi/6}
\end{align*}
We can find the residue at these three points through the formula
\[
	B_k 
	= \Res_{z=c_k} \frac{z^2}{z^6 + 1}
	= \frac{z^2}{6z^5} \Big|_{z=c_k}
	= \frac{1}{6z^3} \Big|_{z=c_k}
	= \frac{1}{6c_k^3}
\]
so we can write
\[
	2\pi i \sum_{k=1}^n B_k
	= 2\pi i \left( \frac{1}{6i} - \frac{1}{6i} + \frac{1}{6i} \right)
	= \frac{\pi}{3}
\]
and, as long as $R > 1$,
\[
	\int_{-R}^R f(x)~dx = \frac{\pi}{3} - \int_{C_R} f(z)~dz
\]

Next, when $|z| = R$, we can write
\[
	\frac{|z^2|}{|z^6 + 1|} < \frac{R^2}{R^6 - 1}
\]
so 
\[
	\left| \int_{C_R} f(z)~dz \right| 
	\le \frac{R^2}{R^6 - 1} \pi R
	= \frac{R^3}{R^6 - 1}
\]
and, as $R \to \infty$, this approaches $0$. Thus,
\[
	\text{P.V.} \int_{-\infty}^\infty f(z)~dz
	= \frac{\pi}{3}
\]
so
\[
	\int_0^\infty \frac{x^2}{x^6 + 1} = \frac{\pi}{6}
\]

\end{document}