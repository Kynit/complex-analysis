\documentclass{article}

\usepackage{amsmath}
\usepackage{parskip}
\renewcommand{\emph}{\textbf}

\begin{document}

\section{Frame 12 -- Functions of Complex Variables}
\subsection{Functions}
If $S$ is a set of complex numbers, then a \emph{function} $f$ is a rule that assigns a complex number $w$ to each $z$ in $S$. The number $w$ is called the \emph{value} of $f$ at $z$. We denote it as
\[
	w = f(z)
\]
The set $S$ is called the \emph{domain of definition} of $f$. Note that we need both a rule ($f$) and a domain ($S$) for a function to be well defined.

Suppose that $w = u + iv$ and $z = x + iy$. Then,
\[
	u + iv = f(x + iy)
\]
Then, we can express $f(z)$ as a pair of real functions of $x$ and $y$:
\[
	f(z) = u(x, y) + iv(x, y)
\]
Alternatively, we could use polar coordinates to write
\[
	u + iv = f(re^{i\theta})
\]
so
\[
	f(z) = u(r, \theta) + iv(r, \theta)
\]

\textit{Example: the function $f(z) = z^2$ can be written as
\begin{align*}
	f(x + iy) &= (x + iy)^2 \\
	&= (x^2 - y^2) + i2xy \\
\intertext{so}
	u(x, y) &= x^2 - y^2 \\
	v(x, y) &= 2xy
\end{align*} 
In polar coordinates,
\begin{align*}
	f(x + iy) &= (re^{i\theta})^2 \\
	&= r^2 e^{i2\theta} \\
	&= r^2 \cos 2\theta + ir^2 \sin 2\theta
\intertext{so}
	u(r, \theta) &= r^2 \cos 2\theta \\
	v(r, \theta) &= r^2 \sin 2\theta
\end{align*}
}


\subsection{Real-Valued Functions}
We say that $f$ is a \emph{real-valued function} if $v$ is zero everywhere. 

\textit{Example: one real-valued function is
\[
	f(z) = |z|^2 = x^2 + y^2 + i0
\]
}


\subsection{Polynomials}
If $n$ is a non-negative integer and $a_0, a_1, a_2, \dots, a_n$ are complex numbers with $a_n \neq 0$, then the function
\[
	P(z) = a_0 + a_1 z + a_2 z^2 + \dots + a_n z^n
\]
is a \emph{polynomial} of degree $n$. Note that this sum has a finite number of terms and that the domain of definition is the entire $z$ plane.

As in real numbers, a \emph{rational function} is a quotient of two polynomials:
\[
	R(z) = \frac{P(z)}{Q(z)}
\]
A rational function is defined everywhere that $Q(z) \neq 0$. 


\subsection{Multi-Valued Functions}
A generalization of a function is a rule that assigns more than one value to a point $z$. These \emph{multiple-valued functions} are usually studied by taking one of the possible values at each point and constructing a single-valued function.

\textit{Example: we know that we can write
\[
	z^{1/2} = \pm \sqrt{r} e^{i\theta/2}
\]
where we denoted $-\pi < \theta \leq \pi$ as the \emph{principal value} of $\text{arg} z$. To turn this into a single valued function, we can choose the positive value of $r$ and write
\[
	f(z) = \sqrt{r} e^{i\theta / 2}
\]
Then, $f$ is well-defined on the entire plane.
}


\clearpage
\section{Frame 13 -- Mappings}
\subsection{Definitions}
There is no convenient way to graph the function $w = f(z)$ -- each of these complex numbers are located on a plane instead of a line. Instead, we can draw pairs of corresponding points on separate $z$ and $w$ planes. When we think of $f$ this way, we call it a \emph{mapping} or \emph{transformation}.

If $f$ is defined on the domain of definition $S$, then the \emph{image} of a point $z \in S$ is the point $w = f(z)$. If $T$ is a subset of $S$, then the set of the images of each point in $T$ are called the image of $T$. In particular, the image of the entire domain, $S$, is called the \emph{range} of $f$. The \emph{inverse image} of a point $w$ is the set of points $z$ in $S$ that map to $w$ (possibly zero, one, or many points).

\subsection{Basic transformations}
Using this geometric interpretation, we can describe mappings using terms such as \emph{translation}, \emph{rotation}, and \emph{reflection}. For instance, the mapping
\[
	w = z + 1 = (x + 1) + iy
\]
can be thought of as a translation of each point $z$ one unit to the right. Another example is the rotational mapping
\begin{align*}
	w &= iz \\
\intertext{where, using $i = e^{i\pi/2}$ and $z = re^{i\theta}$, is}
	w &= r e^{i(\theta + \pi/2)}
\end{align*}
or, in other words, a 90$^\circ$ rotation. Finally, the mapping
\[
	w = \bar{z} = x - iy
\]
is a reflection across the real axis. Usually, it is more useful to sketch an image of a curve rather than a single point. 

\subsection{Mapping a curve}
\textit{For an example, consider the mapping $w = z^2$. We showed earlier that this can be written as
\[
	u = x^2 - y^2, \quad v = 2xy
\]
To sketch the image, we will first set $u = c_1$, which requires that
\[
	x^2 - y^2 = c_1, \quad c_1 > 0
\]
which is the equation for a hyperbola. This equation can then be used to solve for the image points:
\[
	u = c_1, \quad v = \pm2y\sqrt{y^2 + c_1}
\]
where the plus-minus is resolved depending on which side the image point is on. Simply put, as $z$ travels up the right-side hyperbola or down the left-side hyperbola, $w$ travels up the vertical line $u = c_1$.
}

\textit{
Next, we can set $v = c_2$, which requires
\[
	2xy = c_2, \quad c_2 > 0
\]
This gives us the image set
\[
	u = x^2 - \frac{c_2^2}{4x^2}, \quad v = c_2
\]
As $x \to \pm\infty$, $u \to \infty$; as $x \to 0$, $u \to -\infty$. Thus, this hyperbola traces out the straight line $v = c_2$ towards the right as $z$ travels towards the left.
}

\subsection{Mapping a region}
We can use some of the details from the previous example to find the image of a region, rather than a single curve.

\textit{Consider the domain $x > 0, y > 0, xy < 1$. This region consists of the upper branches of the hyperbolas
\[
	2xy = c, \quad 0 < c < 2
\]
and we know from the previous example that these hyperbolas map to the straight lines
\[
	v = c
\]
Thus, this region maps to the horizontal strip $0 < v < 2$.
}

\textit{We can also close the domain to contain the curves $x = 0$, $y = 0$, and $xy = 1$. From the function $w = z^2$, we know that the points $(0, y)$ and $(x, 0)$ map to the points $(-y^2, 0)$ and $(x^2, 0)$, so including the two straight lines simply extends the strip to include $v = 0$. Similarly, the hyperbola $xy = 1$ maps to the horizontal line $v = 2$.}

\textit{Simply put, the image of the closed region $x \ge 0$, $y \ge 0$, $xy \le 1$ is the closed region $0 \le v \le 2$.}


\subsection{Mapping with polar coordinates}
Finally, we can use polar coordinates to simplify some mappings. 

\textit{Again, consider the mapping $w = z^2$. If we write $z = re^{i\theta}$, then the image point can be written as
\[
	w = r^2 e^{2i\theta}
\]
Looking at the magnitude of $w$, points on a circle $r = r_0$ are mapped onto a circle $r' = r_0^2$. Also, looking at the argument of $w$, the angle of the image is doubled. This means that the first quadrant, which is defined as
\[
	r \ge 0, \quad 0 \le \theta \le \pi/2
\]
is in a one-to-one mapping with the top plane, $0 \le \theta \le \pi$. Simiarly, the top place is mapped onto the entire complex plane (although this is not one-to-one, since the inverse image of the positive real axis is both real axes).}

Note that any mapping $w = z^n$ for positive integer $n$ has a similar form, where each non-zero point in the $w$ plane is the image of $n$ distinct points in the $z$ plane.


\clearpage
\section{Frame 14 -- Mappings by the Exponential Function}
Now, we will look at the exponential function
\[
	e^z = e^{x + iy} = e^x e^{iy}
\]
We can again look at straight lines and find their images in this mapping.

\textit{Consider the transformation
\[
	w = e^z = \rho e^{i\phi}
\]
where
\[
	p = e^x \quad \phi = y
\]
This means that the image of a vertical line $x = c_1$ is a circle with radius $p = e^{c_1}$. Each point on the circle is the image of infinitely many points, each spaced $2\pi$ units apart on the vertical line. Similarly, the horizontal line $y = c_2$ is a ray with an angle of $\phi = c_2$.}

With these images in mind, we know that vertical and horizontal line segments are mapped onto arcs and rays, respectively. We can then use this information to map regions:

\textit{Now, consider the rectangular region
\[
	a \le x \le b	\quad	c \le y \le d
\]
The image of this region under the mapping $w = e^z$ is
\[
	e^a \le \rho \le e^b	\quad	c \le \phi \le d
\]
This is a one-to-one mapping if $d - c < 2\pi$. In particular, the region with $c = 0, d = \pi$ is mapped onto half of a circular ring.}



\end{document}