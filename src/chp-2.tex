\documentclass{article}

\usepackage{amsmath}
\usepackage{parskip}
\renewcommand{\emph}{\textbf}

\begin{document}

\section{Frame 12 -- Functions of Complex Variables}
\subsection{Functions}
If $S$ is a set of complex numbers, then a \emph{function} $f$ is a rule that assigns a complex number $w$ to each $z$ in $S$. The number $w$ is called the \emph{value} of $f$ at $z$. We denote it as
\[
	w = f(z)
\]
The set $S$ is called the \emph{domain of definition} of $f$. Note that we need both a rule ($f$) and a domain ($S$) for a function to be well defined.

Suppose that $w = u + iv$ and $z = x + iy$. Then,
\[
	u + iv = f(x + iy)
\]
Then, we can express $f(z)$ as a pair of real functions of $x$ and $y$:
\[
	f(z) = u(x, y) + iv(x, y)
\]
Alternatively, we could use polar coordinates to write
\[
	u + iv = f(re^{i\theta})
\]
so
\[
	f(z) = u(r, \theta) + iv(r, \theta)
\]

\textit{Example: the function $f(z) = z^2$ can be written as
\begin{align*}
	f(x + iy) &= (x + iy)^2 \\
	&= (x^2 - y^2) + i2xy \\
\intertext{so}
	u(x, y) &= x^2 - y^2 \\
	v(x, y) &= 2xy
\end{align*} 
In polar coordinates,
\begin{align*}
	f(x + iy) &= (re^{i\theta})^2 \\
	&= r^2 e^{i2\theta} \\
	&= r^2 \cos 2\theta + ir^2 \sin 2\theta
\intertext{so}
	u(r, \theta) &= r^2 \cos 2\theta \\
	v(r, \theta) &= r^2 \sin 2\theta
\end{align*}
}


\subsection{Real-Valued Functions}
We say that $f$ is a \emph{real-valued function} if $v$ is zero everywhere. 

\textit{Example: one real-valued function is
\[
	f(z) = |z|^2 = x^2 + y^2 + i0
\]
}


\subsection{Polynomials}
If $n$ is a non-negative integer and $a_0, a_1, a_2, \dots, a_n$ are complex numbers with $a_n \neq 0$, then the function
\[
	P(z) = a_0 + a_1 z + a_2 z^2 + \dots + a_n z^n
\]
is a \emph{polynomial} of degree $n$. Note that this sum has a finite number of terms and that the domain of definition is the entire $z$ plane.

As in real numbers, a \emph{rational function} is a quotient of two polynomials:
\[
	R(z) = \frac{P(z)}{Q(z)}
\]
A rational function is defined everywhere that $Q(z) \neq 0$. 


\subsection{Multi-Valued Functions}
A generalization of a function is a rule that assigns more than one value to a point $z$. These \emph{multiple-valued functions} are usually studied by taking one of the possible values at each point and constructing a single-valued function.

\textit{Example: we know that we can write
\[
	z^{1/2} = \pm \sqrt{r} e^{i\theta/2}
\]
where we denoted $-\pi < \theta \leq \pi$ as the \emph{principal value} of $\text{arg} z$. To turn this into a single valued function, we can choose the positive value of $r$ and write
\[
	f(z) = \sqrt{r} e^{i\theta / 2}
\]
Then, $f$ is well-defined on the entire plane.
}


\clearpage
\section{Frame 13 -- Mappings}
\subsection{Definitions}
There is no convenient way to graph the function $w = f(z)$ -- each of these complex numbers are located on a plane instead of a line. Instead, we can draw pairs of corresponding points on separate $z$ and $w$ planes. When we think of $f$ this way, we call it a \emph{mapping} or \emph{transformation}.

If $f$ is defined on the domain of definition $S$, then the \emph{image} of a point $z \in S$ is the point $w = f(z)$. If $T$ is a subset of $S$, then the set of the images of each point in $T$ are called the image of $T$. In particular, the image of the entire domain, $S$, is called the \emph{range} of $f$. The \emph{inverse image} of a point $w$ is the set of points $z$ in $S$ that map to $w$ (possibly zero, one, or many points).

\subsection{Basic transformations}
Using this geometric interpretation, we can describe mappings using terms such as \emph{translation}, \emph{rotation}, and \emph{reflection}. For instance, the mapping
\[
	w = z + 1 = (x + 1) + iy
\]
can be thought of as a translation of each point $z$ one unit to the right. Another example is the rotational mapping
\begin{align*}
	w &= iz \\
\intertext{where, using $i = e^{i\pi/2}$ and $z = re^{i\theta}$, is}
	w &= r e^{i(\theta + \pi/2)}
\end{align*}
or, in other words, a 90$^\circ$ rotation. Finally, the mapping
\[
	w = \bar{z} = x - iy
\]
is a reflection across the real axis. Usually, it is more useful to sketch an image of a curve rather than a single point. 

\subsection{Mapping a curve}
\textit{For an example, consider the mapping $w = z^2$. We showed earlier that this can be written as
\[
	u = x^2 - y^2, \quad v = 2xy
\]
To sketch the image, we will first set $u = c_1$, which requires that
\[
	x^2 - y^2 = c_1, \quad c_1 > 0
\]
which is the equation for a hyperbola. This equation can then be used to solve for the image points:
\[
	u = c_1, \quad v = \pm2y\sqrt{y^2 + c_1}
\]
where the plus-minus is resolved depending on which side the image point is on. Simply put, as $z$ travels up the right-side hyperbola or down the left-side hyperbola, $w$ travels up the vertical line $u = c_1$.
}

\textit{
Next, we can set $v = c_2$, which requires
\[
	2xy = c_2, \quad c_2 > 0
\]
This gives us the image set
\[
	u = x^2 - \frac{c_2^2}{4x^2}, \quad v = c_2
\]
As $x \to \pm\infty$, $u \to \infty$; as $x \to 0$, $u \to -\infty$. Thus, this hyperbola traces out the straight line $v = c_2$ towards the right as $z$ travels towards the left.
}

\subsection{Mapping a region}
We can use some of the details from the previous example to find the image of a region, rather than a single curve.

\textit{Consider the domain $x > 0, y > 0, xy < 1$. This region consists of the upper branches of the hyperbolas
\[
	2xy = c, \quad 0 < c < 2
\]
and we know from the previous example that these hyperbolas map to the straight lines
\[
	v = c
\]
Thus, this region maps to the horizontal strip $0 < v < 2$.
}

\textit{We can also close the domain to contain the curves $x = 0$, $y = 0$, and $xy = 1$. From the function $w = z^2$, we know that the points $(0, y)$ and $(x, 0)$ map to the points $(-y^2, 0)$ and $(x^2, 0)$, so including the two straight lines simply extends the strip to include $v = 0$. Similarly, the hyperbola $xy = 1$ maps to the horizontal line $v = 2$.}

\textit{Simply put, the image of the closed region $x \ge 0$, $y \ge 0$, $xy \le 1$ is the closed region $0 \le v \le 2$.}


\subsection{Mapping with polar coordinates}
Finally, we can use polar coordinates to simplify some mappings. 

\textit{Again, consider the mapping $w = z^2$. If we write $z = re^{i\theta}$, then the image point can be written as
\[
	w = r^2 e^{2i\theta}
\]
Looking at the magnitude of $w$, points on a circle $r = r_0$ are mapped onto a circle $r' = r_0^2$. Also, looking at the argument of $w$, the angle of the image is doubled. This means that the first quadrant, which is defined as
\[
	r \ge 0, \quad 0 \le \theta \le \pi/2
\]
is in a one-to-one mapping with the top plane, $0 \le \theta \le \pi$. Simiarly, the top place is mapped onto the entire complex plane (although this is not one-to-one, since the inverse image of the positive real axis is both real axes).}

Note that any mapping $w = z^n$ for positive integer $n$ has a similar form, where each non-zero point in the $w$ plane is the image of $n$ distinct points in the $z$ plane.


\clearpage
\section{Frame 14 -- Mappings by the Exponential Function}
Now, we will look at the exponential function
\[
	e^z = e^{x + iy} = e^x e^{iy}
\]
We can again look at straight lines and find their images in this mapping.

\textit{Consider the transformation
\[
	w = e^z = \rho e^{i\phi}
\]
where
\[
	p = e^x \quad \phi = y
\]
This means that the image of a vertical line $x = c_1$ is a circle with radius $p = e^{c_1}$. Each point on the circle is the image of infinitely many points, each spaced $2\pi$ units apart on the vertical line. Similarly, the horizontal line $y = c_2$ is a ray with an angle of $\phi = c_2$.}

With these images in mind, we know that vertical and horizontal line segments are mapped onto arcs and rays, respectively. We can then use this information to map regions:

\textit{Now, consider the rectangular region
\[
	a \le x \le b	\quad	c \le y \le d
\]
The image of this region under the mapping $w = e^z$ is
\[
	e^a \le \rho \le e^b	\quad	c \le \phi \le d
\]
This is a one-to-one mapping if $d - c < 2\pi$. In particular, the region with $c = 0, d = \pi$ is mapped onto half of a circular ring.}


\clearpage
\section{Frame 15 -- Limits}
\subsection{Definitions}
Suppose that a function $f$ is defined at all points $z$ in some deleted neighborhood of $z_0$. The statement that the number $w_0$ is the \emph{limit} of $f(z)$ as $z$ approaches $z_0$ means that the point $w = f(z)$ can be made \textit{arbitrarily close} to $w_0$ if we choose $z$ close enough to $z_0$. We write this as
\[
	\lim_{z \to z_0} f(z) = w_0
\]

To be more precise, if this limit exists, then for each positive number $\epsilon$, there is a positive number $\delta$ such that
\[
	|f(z) - w_0| < \epsilon \text{ whenever } 0 < |z - z_0| < \delta
\]

Geometrically, this definition says that each $\epsilon$ neighbourhood around $w_0$ has a corresponding deleted $\delta$ neighbourhood around $z_0$ such that the image of each point in the $\delta$ neighbourhood maps to a point in the $\epsilon$ neighbourhood.

Note that the deleted neighbourhood will always exist if $z_0$ is internal to the domain of definition of $f$. We can extend the definition of a limit to include boundary points by ignoring all of the neighbourhood's points that are outside the domain.

Also note that this definition only allows a given point to be tested as a limit -- it does not provide a method for finding the limit. This will be covered in the next section.

\subsection{Uniqueness}
If the limit of a function $f(z)$ exists at $z_0$, it must be unique. To show this, consider two limits:
\[
	\lim_{z \to z_0} f(z) = w_0 \text{ and } \lim_{z \to z_0} f(z) = w_1
\]
This implies that we can find $\delta_0$ and $\delta_1$ such that
\[
	|f(z) - w_0| < \epsilon \text{ whenever } 0 < |z - z_0| < \delta_0
\]
and
\[
	|f(z) - w_1| < \epsilon \text{ whenever } 0 < |z - z_0| < \delta_1
\]
Now, suppose that $\delta$ is a positive number smaller than both $\delta_0$ and $\delta_1$. Then, for all $0 < |z - z_0| < \delta$, we find that the difference between the two limits is
\begin{align*}
	|w_1 - w_0| &= ||f(z) - w_0| - |f(z) - w_1|| \\
	& \le |f(z) - w_0| + |f(z) - w_1| \\
	& < \epsilon + \epsilon \\
	& = 2\epsilon
\end{align*}
and since $\epsilon$ can be made arbitrarily small, we must have
\[
	w_1 = w_0
\] 


\subsection{Example -- basic limit}
\textit{
Consider the function $f(z) = \frac{i \bar{z}}{2}$. We can show that the limit of this function as $z \to 1$ is
\[
	\lim_{z \to 1} f(z) = \frac{i}{2}
\]
To do this, we observe that
\begin{align*}
	\left| f(z) - \frac{i}{2} \right| 
	&= \left| \frac{i \bar{z}}{2} - \frac{i}{2} \right| \\
	&= \frac{|z - 1|}{2}
\end{align*}
Then, we can fulfill the limit definition by writing
\[
	\left| f(z) - \frac{i}{2} \right| < \epsilon \text{ whenever } \left| z - 1\right| < 2\epsilon
\]
}

\subsection{Example -- direction dependence}
In order for $w_0$ to be a limit of $f$ at $z_0$, the limit conditions must hold if $z$ approaches $z_0$ in any arbitrary manner. 

\textit{Consider the function
\[
	f(z) = \frac{z}{\bar{z}}
\] 
Then, the limit
\[
	\lim_{z \to 0} f(z)
\]
does not exist. To illustrate this, the function's value for any non-zero point $z = (x, 0)$ is
\[
	f(x, 0) = \frac{x + i0}{x - i0} = 1
\]
but the value for any non-zero point $z = (0, y)$ is
\[
	f(0, y) = \frac{0 + iy}{0 - iy} = -1
\]
so the limit would not be unique.
}


\clearpage
\section{Frame 16 -- Theorems on Limits}
Next, it is helpful to connect limits of complex functions and real-valued functions, allowing us to use our knowledge of calculus to simplify the process of finding complex limits

\subsection{Splitting into real functions}
First, the following theorem is helpful:

\textbf{Theorem 1.} Suppose that
\[
	f(z) = u(x, y) + iv(x, y)
\]
and
\[
	z_0 = x_0 + iy_0, \quad	w_0 = u_0 + iv_0
\]
Then, the limit
\[
	\lim_{z \to z_0} = w_0
\]
holds iff
\[
	\lim_{(x,y) \to (x_0, y_0)} u(x, y) = u_0 \text{ and } 
	\lim_{(x,y) \to (x_0, y_0)} v(x, y) = v_0
\]
The two implications of this theorem can be proved by considering the definitions of the neighbourhoods as open disks.

\subsection{Combining simple limits}
\textbf{Theorem 2.} Suppose that
\[
	\lim_{z \to z_0} f(z) = w_0 \text{ and } \lim_{z \to z_0} F(z) = W_0
\]
Then, we can write the following three limits:
\begin{align*}
	\lim_{z \to z_0} f(z) + F(z) &= w_0 + W_0 \\
	\lim_{z \to z_0} f(z) F(z) &= w_0 W_0 \\
	\lim_{z \to z_0} \frac{f(z)}{F(z)} &= \frac{w_0}{W_0} \text{ if } W_0 \neq 0
\end{align*}
These can be proved easily by applying Theorem 1 to each limit.

\subsection{Polynomials}
Using the basic limit definition from the previous section, it is simple to show that
\[
	\lim_{z \to z_0} c = c
\]
and
\[
	\lim_{z \to z_0} z = z_0
\]
for any complex numbers $c$ and $z_0$. Then, by the multiplication property,
\[
	\lim{z \to z_0} z^n = z_0^n
\]
for any positive integer $z$. These limits can be used to show that, for any polynomial
\[
	P(z) = a_0 + a_1 z + a_2 z^2 + \dots + a_n z^n
\]
the limit as $z$ approaches a point $z_0$ is the polynomial's value:
\[
	\lim_{z \to z_0} P(z) = P(z_0)
\]


\clearpage
\section{Frame 17 -- Limits Involving Infinity}
\subsection{The point at infinity}
Sometimes, it is useful to include the \emph{point at infinity} with the complex plane. This point is denoted by $\infty$. In order to visualize it, the complex plane can be drawn with a unit sphere centered at the origin. Then, a line can be drawn from the top of the sphere (or the \textit{north pole}, denoted by $N$) to any point on the plane; the line will pass through exactly one other point $P$ on the sphere. This correspondence (between points on the plane, $z$, and the sphere, $P$) is called a \emph{stereographic projection}, and the sphere is known as the \emph{Riemann sphere}.

No point in the plane corresponds to the point $N$. We can let $N$ correspond to the point at infinity, giving us a one-to-one mapping between points on the sphere and points in the extended complex plane. 

We will make the distinction that a point $z$ is a point in the finite plane unless we specifically describe the point at infinity -- we will specifically mention $\infty$.

\subsection{Neighbourhoods around infinity}
Next, we can define neighbourhoods around the point at infinity. Looking at the Riemann sphere, we notice that all of the points $P$ in the upper hemisphere project to points $z$ outside of the unit disk. 

Further, if $\epsilon$ is a small, positive number, then points in the plane such that
\[
	|z| > \frac{1}{\epsilon}
\]
correspond to points on the sphere close to $N$. Thus, we call the set $|z| > 1/\epsilon$ an \emph{($\epsilon$) neighbourhood} of $\infty$.

\subsection{Limits with infinity}
With this new point at infinity, we can give meaning to the statement
\[
	\lim_{z \to z_0} f(z) = w_0
\]
when $z_0$ or $w_0$ are infinity. We can then use the following theorems:
\begin{align*}
	\lim_{z \to z_0} f(z) = \infty &\iff \lim_{z \to z_0} \frac{1}{f(z)} = 0 \\
	\lim_{z \to \infty} f(z) = w_0 &\iff \lim_{z \to 0} f\Big(\frac{1}{z}\Big) = w_0 \\
	\lim_{z \to \infty} f(z) = \infty &\iff \lim_{z \to 0} \frac{1}{f(1/z)} = 0
\end{align*}

\subsection{Examples}
Three limits using these new properties follow.
\begin{itemize}
\item To find
\[
	\lim_{z \to -1} \frac{iz + 3}{z + 1}
\]
we notice that
\[
	\lim_{z \to -1} \frac{z + 1}{iz + 3} = 0
\]
so the limit is infinity.

\item To find
\[
	\lim_{z \to \infty} \frac{2z + i}{z + 1}
\]
we evaluate
\begin{align*}
	\lim_{z \to 0} \frac{(2 / z) + i}{(1 / z) + 1} 
	&= \lim_{z \to 0} \frac{2 + iz}{1 + z} \\
	&= 2
\end{align*}

\item To find
\[
	\lim_{z \to \infty} \frac{2z^3 - 1}{z^2 + 1}
\]
we evaluate
\begin{align*}
	\lim_{z \to 0} \frac{(1/z^2) + 1}{(2/z^3) - 1}
	&= \lim_{z \to 0} \frac{z + z^3}{2 - z^3} \\
	&= 0
\end{align*}
so the original limit is infinity.
\end{itemize}


\clearpage
\section{Continuity}
\subsection{Definitions}
A function is \emph{continuous} at a point $z_0$ if all of the three following conditions are true:
\begin{align*}
	&\lim_{z \to z_0} f(z) \text{ exists} \\
	&f(z_0) \text{ exists} \\
	&\lim_{z \to z_0} f(z) = f(z_0)
\end{align*}
This final statement says that for each positive number $\epsilon$ there is a positive number $\delta$ such that
\[
	|f(z) - f(z_0)| < \epsilon \text{ whenever } |z - z_0| < \delta
\]

We say that a function is said to be continuous in a region $R$ if it is continuous at each point in $R$. 

\subsection{Theorems}
The basic limit identities allow us to find the continuity of more complex functions. If two functions are continuous at a point, their sum, product, and quotients are also continuous (in the last case, provided that the denominator is non-zero). A polynomial is continuous in the entire plane.

A complex function that can be split into its real and imaginary components
\[
	f(z) = u(x, y) + iv(x, y)
\]
is continuous at $z_0 = (x_0, y_0)$ iff $u$ and $v$ are continuous at $z_0$.

We can state three more theorems about continuity:
\begin{itemize}
	\item A composition of continuous functions is continuous.
	
	\textit{Suppose that $w = f(z)$ is defined in a neighbourhood of $z_0$ and $W = g(w)$ is defined in a neighbourhood of $f(z_0)$. Also, suppose that $f$ is continuous at $z_0$ and $g$ is continuous at $f(z_0)$. Then, the statement that the composition
	\[
		g[f(z)]
	\]
	is continuous is equivalent to the statement that
	\[
		|g[f(z)] - g[f(z_0)]| < \epsilon  \text{ whenever } | f(z) - f(z_0)| < \gamma
	\]
	Then, since $f$ is continuous, we can find a $\delta$ such that the right side is satisfied, so $g \circ f$ is continuous.
	}
	
	\item If $f$ is continuous and non-zero at $z_0$, then there is some neighbourhood of $z_0$ where $f(z) \neq 0$.
	
	\textit{Suppose that we choose $\epsilon = |f(z_0)| / 2$. Then, if there is a point where $f(z) = 0$ in a $\delta$ neighbourhood around $z_0$, the limit inequality is
	\[
		|f(z_0)| < \frac{|f(z_0)|}{2}
	\]
	so we have a contradiction, and there must be a neighbourhood where $f(z) = 0$.
	}
	
	\item If $f$ is continuous in a closed and bounded region $R$, then there exists a non-negative real number $M$ such that
	\[
		|f(z)| \le M
	\]
	where equality holds for one or more $z$.
\end{itemize}



\end{document}