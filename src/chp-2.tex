\documentclass{article}

\usepackage{amsmath}
\usepackage{parskip}
\renewcommand{\emph}{\textbf}

\begin{document}

\section{Frame 12 -- Functions of Complex Variables}
\subsection{Functions}
If $S$ is a set of complex numbers, then a \emph{function} $f$ is a rule that assigns a complex number $w$ to each $z$ in $S$. The number $w$ is called the \emph{value} of $f$ at $z$. We denote it as
\[
	w = f(z)
\]
The set $S$ is called the \emph{domain of definition} of $f$. Note that we need both a rule ($f$) and a domain ($S$) for a function to be well defined.

Suppose that $w = u + iv$ and $z = x + iy$. Then,
\[
	u + iv = f(x + iy)
\]
Then, we can express $f(z)$ as a pair of real functions of $x$ and $y$:
\[
	f(z) = u(x, y) + iv(x, y)
\]
Alternatively, we could use polar coordinates to write
\[
	u + iv = f(re^{i\theta})
\]
so
\[
	f(z) = u(r, \theta) + iv(r, \theta)
\]

\textit{Example: the function $f(z) = z^2$ can be written as
\begin{align*}
	f(x + iy) &= (x + iy)^2 \\
	&= (x^2 - y^2) + i2xy \\
\intertext{so}
	u(x, y) &= x^2 - y^2 \\
	v(x, y) &= 2xy
\end{align*} 
In polar coordinates,
\begin{align*}
	f(x + iy) &= (re^{i\theta})^2 \\
	&= r^2 e^{i2\theta} \\
	&= r^2 \cos 2\theta + ir^2 \sin 2\theta
\intertext{so}
	u(r, \theta) &= r^2 \cos 2\theta \\
	v(r, \theta) &= r^2 \sin 2\theta
\end{align*}
}


\subsection{Real-Valued Functions}
We say that $f$ is a \emph{real-valued function} if $v$ is zero everywhere. 

\textit{Example: one real-valued function is
\[
	f(z) = |z|^2 = x^2 + y^2 + i0
\]
}


\subsection{Polynomials}
If $n$ is a non-negative integer and $a_0, a_1, a_2, \dots, a_n$ are complex numbers with $a_n \neq 0$, then the function
\[
	P(z) = a_0 + a_1 z + a_2 z^2 + \dots + a_n z^n
\]
is a \emph{polynomial} of degree $n$. Note that this sum has a finite number of terms and that the domain of definition is the entire $z$ plane.

As in real numbers, a \emph{rational function} is a quotient of two polynomials:
\[
	R(z) = \frac{P(z)}{Q(z)}
\]
A rational function is defined everywhere that $Q(z) \neq 0$. 


\subsection{Multi-Valued Functions}
A generalization of a function is a rule that assigns more than one value to a point $z$. These \emph{multiple-valued functions} are usually studied by taking one of the possible values at each point and constructing a single-valued function.

\textit{Example: we know that we can write
\[
	z^{1/2} = \pm \sqrt{r} e^{i\theta/2}
\]
where we denoted $-\pi < \theta \leq \pi$ as the \emph{principal value} of $\text{arg} z$. To turn this into a single valued function, we can choose the positive value of $r$ and write
\[
	f(z) = \sqrt{r} e^{i\theta / 2}
\]
Then, $f$ is well-defined on the entire plane.
}

\end{document}